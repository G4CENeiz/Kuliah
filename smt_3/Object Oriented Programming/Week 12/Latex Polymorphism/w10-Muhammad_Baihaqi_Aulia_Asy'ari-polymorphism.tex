\documentclass[12pt,titlepage]{article}
\usepackage[margin=1.25in]{geometry}
\usepackage{graphicx,amsmath,blindtext,minted}

%% Variables definition
\newcommand{\vSubject}{Object Oriented Programming}
\newcommand{\vSubtitle}{Polymorphism}
\newcommand{\vName}{Muhammad Baihaqi Aulia Asy'ari}
\newcommand{\vNIM}{2241720145}
\newcommand{\vClass}{2I}
\newcommand{\vDepartment}{Information Technology}
\newcommand{\vStudyProgram}{D4 Informatics Engineering}

%% [START] Tikz related stuff
\usepackage{tikz}
\usetikzlibrary{svg.path,calc,shapes.geometric,shapes.misc}
\tikzstyle{terminator} = [rectangle, draw, text centered, rounded corners = 1em, minimum height=2em]
\tikzstyle{preparation} = [chamfered rectangle, chamfered rectangle sep=0.75em, draw, text centered, minimum height = 2em]
\tikzstyle{process} = [rectangle, draw, text centered, minimum height=2em]
\tikzstyle{decision} = [diamond, aspect=2, draw, text centered, minimum height=2em]
\tikzstyle{data}=[trapezium, draw, text centered, trapezium left angle=60, trapezium right angle=120, minimum height=2em]
\tikzstyle{connector} = [line width=0.25mm,->]
%% [END] Tikz related stuff

%% [START] Fancy header related stuff
\usepackage{fancyhdr}
\pagestyle{fancy}
\setlength{\headheight}{15pt} % compensate fancyhdr style
\fancyhead{}
\fancyfoot{}
\fancyfoot[L]{\thepage}
\fancyfoot[R]{\textit{\vSubject - \vSubtitle}}
\renewcommand{\footrulewidth}{0.4pt}% default is 0pt, overline for footer
%% [END] Fancy header related stuff

%% [START] Custom tabular command related stuff
\usepackage{tabularx}
\newcommand{\details}[2]{
    #1 & #2  \\
}
%% [END] Custom tabular command related stuff

%% [START] Figure related stuff
\newcommand{\image}[3][1]{
    \begin{figure}[h]
        \centering
        \includegraphics[#1]{#2}
        \caption{#3}
        \label{#3}
    \end{figure}
}
%% [END] Figure related stuff

%%
\usepackage{pgf-umlcd}

\renewcommand{\umldrawcolor}{black}
\renewcommand{\umlfillcolor}{white}
%%

%% [BEGIN] Custom enumerator
\usepackage{enumitem}
%% [END] Custom enumerator

%% [BEGIN] Paragraph indent
\usepackage{indentfirst}
%% [END] Paragraph indent

\begin{document}
\begin{titlepage}
    \centering
    \vfill
    {\bfseries\LARGE
        \vSubject\\
        \vskip0.25cm
        \vSubtitle
    }
    \vfill
    \includegraphics[width=6cm]{images/polinema-logo.png}
    \vfill
    {
        \textbf{Name}\\
        \vName\\
        \vskip0.5cm
        \textbf{NIM}\\
        \vNIM\\
        \vskip0.5cm
        \textbf{Class}\\
        \vClass\\
        \vskip0.5cm
        \textbf{Department}\\
        \vDepartment\\
        \vskip0.5cm
        \textbf{Study Program}\\
        \vStudyProgram
    }
\end{titlepage}

\newpage

\setcounter{section}{3}

\section{Experiment 1 - Basic Form of Polymorphism}
\subsection{Experiment Step}

\texttt{Employee.java}
\begin{minted}[autogobble,breaklines,linenos]{java}
    package polymorphism.experiment1;

    public class Employee {
        protected String name;

        public String getEmployeeInfo() {
            return "Name = " + name;
        }
    }
\end{minted}

\texttt{Payable.java}
\begin{minted}[autogobble,breaklines,linenos]{java}
    package polymorphism.experiment1;

    public interface Payable {
        public int getPaymentAmount();
    }
\end{minted}

\texttt{InternshipEmployee.java}
\begin{minted}[autogobble,breaklines,linenos]{java}
    package polymorphism.experiment1;

    public class InternshipEmployee extends Employee {
        private int length;

        public InternshipEmployee(String name, int length) {
            this.length = length;
            this.name = name;
        }

        public int getLength() {
            return length;
        }

        public void setLength(int length) {
            this.length = length;
        }

        @Override
        public String getEmployeeInfo() {
            String info = super.getEmployeeInfo() + "\n";
            info += "Registered as internship employee for " + length + " month/s\n";
            return info;
        }
    }
\end{minted}

\texttt{PermanentEmployee.java}
\begin{minted}[autogobble,breaklines,linenos]{java}
    package polymorphism.experiment1;

    public class PermanentEmployee extends Employee implements Payable {
        private int salary;

        public PermanentEmployee(String name, int salary) {
            this.name = name;
            this.salary = salary;
        }

        public int getSalary() {
            return salary;
        }

        public void setSalary(int salary) {
            this.salary = salary;
        }

        @Override
        public int getPaymentAmount() {
            return (int) (salary + 0.05 * salary);
        }

        @Override
        public String getEmployeeInfo() {
            String info = super.getEmployeeInfo() + "\n";
            info += "Registered as permanent emplyee with salary " + salary + "\n";
            return info;
        }
    }
\end{minted}

\texttt{ElectricityBill.java}
\begin{minted}[autogobble,breaklines,linenos]{java}
    package polymorphism.experiment1;

    public class ElectricityBill implements Payable {
        private int kwh;
        private String category;

        public ElectricityBill(int kwh, String category) {
            this.kwh = kwh;
            this.category = category;
        }

        public int getKwh() {
            return kwh;
        }

        public void setKwh(int kwh) {
            this.kwh = kwh;
        }

        public String getCategory() {
            return category;
        }

        public void setCategory(String category) {
            this.category = category;
        }

        @Override
        public int getPaymentAmount() {
            return kwh + getBasePrice();
        }

        public int getBasePrice() {
            int bPrice = 0;
            switch (category) {
                case "R-1": bPrice = 100;break;
                case "R-2": bPrice = 200;break;
            }
            return bPrice;
        }

        public String getBillInfo() {
            return  "kWH = " + kwh + "\n" +
                    "Category = " + category + "(" + getBasePrice() + " per kWH)\n";
        }
    }
\end{minted}

\texttt{Tester1.java}
\begin{minted}[autogobble,breaklines,linenos]{java}
    package polymorphism.experiment1;

    public class Tester1 {
        public static void main(String[] args) {
            PermanentEmployee pEmp = new PermanentEmployee("Alpha", 420);
            InternshipEmployee iEmp = new InternshipEmployee("Bravo", 6);
            ElectricityBill eBill = new ElectricityBill(5, "A-1");
            Employee e;
            Payable p;
            e = pEmp;
            e = iEmp;
            p = pEmp;
            p = eBill;
        }
    }
\end{minted}

\subsection{Questions}
\begin{enumerate}
    \item Class apa sajakah yang merupakan turunan dari class Employee?
    \item Class apa sajakah yang implements ke interface Payable?
    \item Perhatikan class Tester1, baris ke-10 dan 11. Mengapa e, bisa diisi dengan objek pEmp (merupakan objek dari class PermanentEmployee) dan objek iEmp (merupakan objek dari class InternshipEmploye) ?
    \item Perhatikan class Tester1, baris ke-12 dan 13. Mengapa p, bisa diisi dengan objek pEmp (merupakan objek dari class PermanentEmployee) dan objek eBill (merupakan objek dari class ElectricityBill) ?
    \item Coba tambahkan sintaks:\\
    p = iEmp;\\
    e = eBill;\\
    pada baris 14 dan 15 (baris terakhir dalam method main) ! Apa yang menyebabkan error?
    \item Ambil kesimpulan tentang konsep/bentuk dasar polimorfisme!
\end{enumerate}

\subsection{Answers}
\begin{enumerate}
    \item InternshipEmployee and PermanentEmployee.
    \item PermanentEmployee and ElectricityBill.
    \item Because both are subclass of Employee.
    \item Because both implements the interface Payable.
    \item The iEmp does not implements the interface Payable and the eBill is not a subclass or does not extends the class Employee.
    \item Polymorphism allow an object to take form of the superclass or its interface.
\end{enumerate}

\section{Experiment 2 - Virtual Method Invocation}
\subsection{Experiment Step}
\texttt{Tester2.java}
\begin{minted}[autogobble,breaklines,linenos]{java}
    package polymorphism.experiment1;

    public class Tester2 {
        public static void main(String[] args) {
            PermanentEmployee pEmp = new PermanentEmployee("Alpha", 420);
            Employee e;
            e = pEmp;
            System.out.println("" + e.getEmployeeInfo());
            System.out.println("------------------");
            System.out.println("" + pEmp.getEmployeeInfo());
        }
    }
\end{minted}

\texttt{Terminal}
\begin{minted}[autogobble,breaklines,linenos]{text}
    PS D:\Kuliah>  & 'C:\Program Files\Java\jdk-18.0.2.1\bin\java.exe' '-XX:+ShowCodeDetailsInExceptionMessages' '-cp' 'C:\Users\G4CE-PC\AppData\Roaming\Code\User\workspaceStorage\ 80d97a47d24665dc0bce7ab1e048ecbd\redhat.java\jdt_ws\ Kuliah_28156aa7\bin' 'polymorphism.experiment1.Tester2'     
    Name = Alpha
    Registered as permanent emplyee with salary 420

    ------------------
    Name = Alpha
    Registered as permanent emplyee with salary 420
\end{minted}
\subsection{Questions}
\begin{enumerate}
    \item Perhatikan class Tester2 di atas, mengapa pemanggilan e.getEmployeeInfo() pada baris 8 dan pEmp.getEmployeeInfo() pada baris 10 menghasilkan hasil sama? 
    \item Mengapa pemanggilan method e.getEmployeeInfo() disebut sebagai pemanggilan method virtual (virtual method invication), sedangkan pEmp .getEmployeeInfo() tidak?
    \item Jadi apakah yang dimaksud dari virtual method invocation? Mengapa disebut virtual?
\end{enumerate}
\subsection{Answers}
\begin{enumerate}
    \item Because despite the e is an object of Employee, the method being called still came from the pEmp object that it is declared from.
    \item Because when the e.getEmployeeInfo() is compiled as the Employee method, the JVM recognize that the object e is an object of PermanentEmployee in which it used that method instead of what is already compiled.
    \item it is when the method compiled followed what the Class declared but in the run time used what has been declared by the Object being casted.
\end{enumerate}

\newpage
\section{Experiment 3 - Heterogenous Collection}
\subsection{Experiment Step}
\texttt{Tester3.java}
\begin{minted}[autogobble,breaklines,linenos]{java}
    package polymorphism.experiment1;

    public class Tester3 {
        public static void main(String[] args) {
            PermanentEmployee pEmp = new PermanentEmployee("Alpha", 420);
            InternshipEmployee iEmp = new InternshipEmployee("Bravo", 6);
            ElectricityBill eBill = new ElectricityBill(5, "A-1");
            Employee[] e = {pEmp, iEmp};
            Payable[] p = {pEmp, eBill};
            Employee[] e2 = {pEmp, iEmp, eBill};
        }
    }
\end{minted}

\texttt{Terminal}
\begin{minted}[autogobble,breaklines,linenos]{text}
    PS D:\Kuliah>  d:; cd 'd:\Kuliah'; & 'C:\Program Files\Java\jdk-18.0.2.1\bin\java.exe' '-XX:+ShowCodeDetailsInExceptionMessages' '-cp' 'C:\Users\G4CE-PC\AppData\Roaming\Code\User\workspaceStorage\ 80d97a47d24665dc0bce7ab1e048ecbd\redhat.java\jdt_ws\ Kuliah_28156aa7\bin' 'polymorphism.experiment1.Tester3'
    Exception in thread "main" java.lang.Error: Unresolved compilation problem: 
            Type mismatch: cannot convert from ElectricityBill to Employee      
    
            at polymorphism.experiment1.Tester3.main(Tester3.java:10)
\end{minted}
\subsection{Questions}
\begin{enumerate}
    \item Perhatikan array e pada baris ke-8, mengapa ia bisa diisi dengan objek-objek dengan tipe yang berbeda, yaitu objek pEmp (objek dari PermanentEmployee) dan objek iEmp (objek dari InternshipEmployee) ?
    \item Perhatikan juga baris ke-9, mengapa array p juga biisi dengan objek-objek dengan tipe yang berbeda, yaitu objek pEmp (objek dari PermanentEmployee) dan objek eBill (objek dari ElectricityBilling) ?
    \item Perhatikan baris ke-10, mengapa terjadi error?
\end{enumerate}
\subsection{Answers}
\begin{enumerate}
    \item Because all of them are based on the class Employee.
    \item Because all of them implements the interface Payable.
    \item Because the ElectricityBill is not a subclass of Employee.
\end{enumerate}

\section{Experiment 4 - Polymorphic Arguments, \\ instanceof and object casting}
\subsection{Experiment Step}
\texttt{Owner.java}
\begin{minted}[autogobble,breaklines,linenos]{java}
    package polymorphism.experiment1;

    public class Owner {
        public void pay(Payable p) {
            System.out.println("Total payment = " + p.getPaymentAmount());
            if (p instanceof ElectricityBill) {
                ElectricityBill eb = (ElectricityBill) p;
                System.out.println("" + eb.getBillInfo());
            } else if (p instanceof PermanentEmployee) {
                PermanentEmployee pe = (PermanentEmployee) p;
                pe.getEmployeeInfo();
                System.out.println("" + pe.getEmployeeInfo());
            }
        }

        public void showMyEmployee(Employee e) {
            System.out.println("" + e.getEmployeeInfo());
            if (e instanceof PermanentEmployee) {
                System.out.println("You have to pay her/him monthly!!!");
            } else {
                System.out.println("No need to pay him/her :)");
            }
        }
    }
\end{minted}

\texttt{Tester4.java}
\begin{minted}[autogobble,breaklines,linenos]{java}
    package polymorphism.experiment1;

    public class Tester4 {
        public static void main(String[] args) {
            Owner ow = new Owner();
            ElectricityBill eBill = new ElectricityBill(5, "R-1");
            ow.pay(eBill);
            System.out.println("------------------------------");
            
            PermanentEmployee pEmp = new PermanentEmployee("Alpha", 420);
            ow.pay(pEmp);
            System.out.println("------------------------------");
            
            InternshipEmployee iEmp = new InternshipEmployee("Bravo", 6);
            ow.showMyEmployee(pEmp);
            System.out.println("------------------------------");
            ow.showMyEmployee(iEmp);
        }
    }
\end{minted}
\subsection{Questions}
\begin{enumerate}
    \item Perhatikan class Tester4 baris ke-7 dan baris ke-11, mengapa pemanggilan ow.pay(eBill) dan ow.pay(pEmp) bisa dilakukan, padahal jika diperhatikan method pay() yang ada di dalam class Owner memiliki argument/parameter bertipe Payable? Jika diperhatikan lebih detil eBill merupakan objek dari ElectricityBill dan pEmp merupakan objek dari PermanentEmployee?
    \item Jadi apakah tujuan membuat argument bertipe Payable pada method pay() yang ada di dalam class Owner?
    \item Coba pada baris terakhir method main() yang ada di dalam class Tester4 ditambahkan perintah ow.pay(iEmp);
    \begin{minted}[autogobble,breaklines,linenos]{java}
        package polymorphism.experiment1;
    
        public class Tester4 {
            public static void main(String[] args) {
                Owner ow = new Owner();
                ElectricityBill eBill = new ElectricityBill(5, "R-1");
                ow.pay(eBill);
                System.out.println("------------------------------");
                
                PermanentEmployee pEmp = new PermanentEmployee("Alpha", 420);
                ow.pay(pEmp);
                System.out.println("------------------------------");
                
                InternshipEmployee iEmp = new InternshipEmployee("Bravo", 6);
                ow.showMyEmployee(pEmp);
                System.out.println("------------------------------");
                ow.showMyEmployee(iEmp);

                ow.pay(iEmp);
            }
        }
    \end{minted}
    Mengapa terjadi error?
    \item Perhatikan class Owner, diperlukan untuk apakah sintaks p instanceof ElectricityBill pada baris ke-6 ?
    \item Perhatikan kembali class Owner baris ke-7, untuk apakah casting objek disana (ElectricityBill eb = (ElectricityBill) p) diperlukan ? Mengapa objek p yang bertipe Payable harus di-casting ke dalam objek eb yang bertipe ElectricityBill ?
\end{enumerate}
\subsection{Answers}
\begin{enumerate}
    \item Correct assessment, but those classes also implements the interface Payable which made them pass-able as an argument fot the method pay.
    \item So that only class that implements the interface Payable can use the method pay().
    \item Because the iEmp object are not based on a class that implements the interface Payable.
    \item It is used to identify whether or not the object p is an instance of ElectricityBill.
    \item So that the object p can use the method of its instance.
\end{enumerate}

\newpage

\section{Assignment}

\texttt{Destroyable.java}
\begin{minted}[autogobble,breaklines,linenos]{java}
    package polymorphism.assignment;

    public interface Destroyable {
        public void destroyed();
    }
\end{minted}

\texttt{Zombie.java}
\begin{minted}[autogobble,breaklines,linenos]{java}
    package polymorphism.assignment;

    public class Zombie implements Destroyable {
        protected int health;
        protected int level;

        public void heal() {
            
        }

        @Override
        public void destroyed() {
            
        }

        public String getZombieInfo() {
            return "";
        }
    }
\end{minted}

\texttt{Barrier.java}
\begin{minted}[autogobble,breaklines,linenos]{java}
    package polymorphism.assignment;

    public class Barrier implements Destroyable {
        private int strength;

        public Barrier(int strength) {
            this.strength = strength;
        }

        public void setStrength(int strength) {
            this.strength = strength;
        }

        public int getStrength() {
            return strength;
        }

        @Override
        public void destroyed() {
            strength -= 9;
        }

        public String getBarrierInfo() {
            return String.format("Barrier Strength = %d%n", strength);
        }
    }
\end{minted}

\texttt{WalkingZombie.java}
\begin{minted}[autogobble,breaklines,linenos]{java}
    package polymorphism.assignment;

    public class WalkingZombie extends Zombie {
        public WalkingZombie(int health, int level) {
            this.health = health;
            this.level = level;
        }

        @Override
        public void heal() {
            if (level == 1) {
                health += health * 0.1;
            } else if (level == 2) {
                health += health * 0.3;
            } else if (level == 3) {
                health += health * 0.4;
            }
        }

        @Override
        public void destroyed() {
            health -= Math.floor(health * 0.2);
        }

        @Override
        public String getZombieInfo() {
            String info = "Walking Zombie Data = \n";
            info += String.format("Health = %d %n", health);
            info += String.format("Level = %d %n", level);
            return info;
        }
    }
\end{minted}

\texttt{JumpingZombie.java}
\begin{minted}[autogobble,breaklines,linenos]{java}
    package polymorphism.assignment;

    public class JumpingZombie extends Zombie {
        public JumpingZombie(int health, int level) {
            this.health = health;
            this.level = level;
        }

        @Override
        public void heal() {
            if (level == 1) {
                health += health * 0.3;
            } else if (level == 2) {
                health += health * 0.4;
            } else if (level == 3) {
                health += health * 0.5;
            }
        }

        @Override
        public void destroyed() {
            health -= Math.floor(health * 0.1);
        }

        @Override
        public String getZombieInfo() {
            String info = "Jumping Zombie Data = \n";
            info += String.format("Health = %d %n", health);
            info += String.format("Level = %d %n", level);
            return info;
        }
    }
\end{minted}

\texttt{Plant.java}
\begin{minted}[autogobble,breaklines,linenos]{java}
    package polymorphism.assignment;

    public class Plant {
        public void doDestroy(Destroyable d) {
            if (d instanceof WalkingZombie) {
                WalkingZombie wz = (WalkingZombie) d;
                wz.destroyed();
            } else if (d instanceof JumpingZombie) {
                JumpingZombie jz = (JumpingZombie) d;
                jz.destroyed();
            } else if (d instanceof Barrier) {
                Barrier b = (Barrier) d;
                b.destroyed();
            }
        }
    }
\end{minted}

\texttt{Tester.java}
\begin{minted}[autogobble,breaklines,linenos]{java}
    package polymorphism.assignment;

    public class Tester {
        public static void main(String[] args) {
            WalkingZombie wz = new WalkingZombie(100, 1);
            JumpingZombie jz = new JumpingZombie(100, 2);
            Barrier b = new Barrier(100);
            Plant p = new Plant();

            System.out.println("" + wz.getZombieInfo());
            System.out.println("" + jz.getZombieInfo());
            System.out.println("" + b.getBarrierInfo());
            System.out.println("------------------------");
            for (int i = 0; i < 4; i++) {
                p.doDestroy(wz);
                p.doDestroy(jz);
                p.doDestroy(b);
            }
            System.out.println("" + wz.getZombieInfo());
            System.out.println("" + jz.getZombieInfo());
            System.out.println("" + b.getBarrierInfo());
        }
    }
\end{minted}

\texttt{Terminal}
\begin{minted}[autogobble,breaklines,linenos]{text}
    PS D:\Kuliah>  d:; cd 'd:\Kuliah'; & 'C:\Program Files\Java\jdk-18.0.2.1\bin\java.exe' '-XX:+ShowCodeDetailsInExceptionMessages' '-cp' 'C:\Users\G4CE-PC\AppData\Roaming\Code\User\workspaceStorage\ 80d97a47d24665dc0bce7ab1e048ecbd\redhat.java\jdt_ws\ Kuliah_28156aa7\bin' 'polymorphism.assignment.Tester' 
    Walking Zombie Data = 
    Health = 100
    Level = 1
    
    Jumping Zombie Data = 
    Health = 100
    Level = 2
    
    Barrier Strength = 100
    
    ------------------------
    Walking Zombie Data =
    Health = 42
    Level = 1
    
    Jumping Zombie Data =
    Health = 66
    Level = 2
    
    Barrier Strength = 64
    
\end{minted}

\end{document}