\documentclass[12pt,titlepage]{article}
\usepackage[margin=1.25in]{geometry}
\usepackage{graphicx,amsmath,blindtext,minted}

%% Variables definition
\newcommand{\vSubject}{Object Oriented Programming}
\newcommand{\vSubtitle}{Quiz 2}
\newcommand{\vName}{Muhammad Baihaqi Aulia Asy'ari}
\newcommand{\vNIM}{2241720145}
\newcommand{\vClass}{2I}
\newcommand{\vDepartment}{Information Technology}
\newcommand{\vStudyProgram}{D4 Informatics Engineering}

%% [START] Tikz related stuff
\usepackage{tikz}
\usetikzlibrary{svg.path,calc,shapes.geometric,shapes.misc}
\tikzstyle{terminator} = [rectangle, draw, text centered, rounded corners = 1em, minimum height=2em]
\tikzstyle{preparation} = [chamfered rectangle, chamfered rectangle sep=0.75em, draw, text centered, minimum height = 2em]
\tikzstyle{process} = [rectangle, draw, text centered, minimum height=2em]
\tikzstyle{decision} = [diamond, aspect=2, draw, text centered, minimum height=2em]
\tikzstyle{data}=[trapezium, draw, text centered, trapezium left angle=60, trapezium right angle=120, minimum height=2em]
\tikzstyle{connector} = [line width=0.25mm,->]
%% [END] Tikz related stuff

%% [START] Fancy header related stuff
\usepackage{fancyhdr}
\pagestyle{fancy}
\setlength{\headheight}{15pt} % compensate fancyhdr style
\fancyhead{}
\fancyfoot{}
\fancyfoot[L]{\thepage}
\fancyfoot[R]{\textit{\vSubject - \vSubtitle}}
\renewcommand{\footrulewidth}{0.4pt}% default is 0pt, overline for footer
%% [END] Fancy header related stuff

%% [START] Custom tabular command related stuff
\usepackage{tabularx}
\newcommand{\details}[2]{
    #1 & #2  \\
}
%% [END] Custom tabular command related stuff

%% [START] Figure related stuff
\newcommand{\image}[3][1]{
    \begin{figure}[h]
        \centering
        \includegraphics[#1]{#2}
        \caption{#3}
        \label{#3}
    \end{figure}
}
%% [END] Figure related stuff

%%
\usepackage{pgf-umlcd}

\renewcommand{\umldrawcolor}{black}
\renewcommand{\umlfillcolor}{white}
%%

%% [BEGIN] Custom enumerator
\usepackage{enumitem}
%% [END] Custom enumerator

%% [BEGIN] Paragraph indent
\usepackage{indentfirst}
%% [END] Paragraph indent

\begin{document}
\begin{titlepage}
    \centering
    \vfill
    {\bfseries\LARGE
        \vSubject\\
        \vskip0.25cm
        \vSubtitle
    }
    \vfill
    \includegraphics[width=6cm]{images/polinema-logo.png}
    \vfill
    {
        \textbf{Name}\\
        \vName\\
        \vskip0.5cm
        \textbf{NIM}\\
        \vNIM\\
        \vskip0.5cm
        \textbf{Class}\\
        \vClass\\
        \vskip0.5cm
        \textbf{Department}\\
        \vDepartment\\
        \vskip0.5cm
        \textbf{Study Program}\\
        \vStudyProgram
    }
\end{titlepage}

\newpage

\section{Quiz 2 Object Oriented Programming}
\begin{enumerate}
    \item \texttt{Student.java}
    \begin{minted}[autogobble,breaklines,linenos]{java}
        import java.util.List;

        public class Student {
            // defining class fields
            private String name;
            private int nim;
            private List<Double> grades;

            // defining constructor
            public Student(String name, int nim, List<Double> grades) {
                this.name = name;
                this.nim = nim;
                this.grades = grades;
            }

            // field getter setter
            public void setName(String name) {
                this.name = name;
            }

            public String getName() {
                return name;
            }

            public void setNim(int nim) {
                this.nim = nim;
            }

            public int getNim() {
                return nim;
            }

            public void setGrade(List<Double> grades) {
                this.grades = grades;
            }

            public List<Double> getGrade() {
                return grades;
            }

            // defining method to calculate GPA
            public double calcualteGPA() {
                double sum = 0;
                for (double grade : getGrade()) {
                    sum += grade;
                }
                return sum/getGrade().size();
            }

            // overload method with new param
            public double calcualteGPA(List<Double> credits) {
                double sum = 0;
                for (int i = 0; i < getGrade().size(); i++) {
                    sum += getGrade().get(i) * credits.get(i);
                }
                return sum/getGrade().size();
            }
        }
    \end{minted}
    \item \texttt{Subject.java}
    \begin{minted}[autogobble,breaklines,linenos]{java}
        import java.util.List;

        public class Subject {
            // defining class fields
            private String subjectName;
            private double credit;
            private List<Double> studentsGrade;

            // defining constructor
            public Subject(String subjectName, double credit, List<Double> studentsGrade) {
                this.subjectName = subjectName;
                this.credit = credit;
                this.studentsGrade = studentsGrade;
            }

            // field getter setter
            public void setSubjectName(String subjectName) {
                this.subjectName = subjectName;
            }

            public String getSubjectName() {
                return subjectName;
            }

            public void setCredit(double credit) {
                this.credit = credit;
            }

            public double getCredit() {
                return credit;
            }

            public void setStudentsGrade(List<Double> studentsGrade) {
                this.studentsGrade = studentsGrade;
            }

            public List<Double> getStudentsGrade() {
                return studentsGrade;
            }

            // defining method to calculate weight
            public double calculateWeight() {
                double sum = 0;
                for (int i = 0; i < studentsGrade.size(); i++) {
                    sum += studentsGrade.get(i);
                }
                return sum/studentsGrade.size();
            }
            // overload method with new param
            public double calculateWeight(List<Double> studentsWeight) {
                double sum = 0;
                for (int i = 0; i < studentsGrade.size(); i++) {
                    sum += studentsGrade.get(i) * studentsWeight.get(i);
                }
                return sum/studentsGrade.size();
            }
        }
    \end{minted}
    \item \texttt{Representative.java}
    \begin{minted}[autogobble,breaklines,linenos]{java}
        import java.util.List;

        public class Representative {
            // defining class fields
            private List<Student> students;
            private List<Subject> subjects;

            // field getter setter
            public void setStudents(List<Student> students) {
                this.students = students;
            }

            public List<Student> getStudents() {
                return students;
            }

            public void setSubjects(List<Subject> subjects) {
                this.subjects = subjects;
            }

            public List<Subject> getSubjects() {
                return subjects;
            }

            // define method to list student and subject
            public void displayStudentList() {
                System.out.println("Student and Subject List");
                for (int i = 0; i < getStudents().size(); i++) {
                    String name = getStudents().get(i).getName();
                    double nim = getStudents().get(i).getNim();
                    double grade = getStudents().get(i).calcualteGPA();
                    String subjectName = getSubjects().get(i).getSubjectName();
                    double credit = getSubjects().get(i).getCredit();
                    double weight = getSubjects().get(i).calculateWeight();

                    System.out.printf("Student  : %s %n", name);
                    System.out.printf("  NIM    : %.2f %n", nim);
                    System.out.printf("  GPA    : %.2f %n", grade);
                    System.out.printf("Subject  : %s %n", subjectName);
                    System.out.printf("  Credit : %.2f %n", credit);
                    System.out.printf("  Weight : %.2f %n", weight);
                }
            }
        }
    \end{minted}
    \item \texttt{Main.java}
    \begin{minted}[autogobble,breaklines,linenos]{java}
        import java.util.ArrayList;
        import java.util.List;

        public class Main {
            public static void main(String[] args) {
                // define grades
                List<Double> grades1 = new ArrayList<Double>();
                grades1.add(90.0);
                grades1.add(87.0);
                grades1.add(80.0);
                List<Double> grades2 = new ArrayList<Double>();
                grades2.add(80.0);
                grades2.add(90.0);
                grades2.add(87.0);

                // define students
                Student student1 = new Student("student 1", 0001, grades1);
                Student student2 = new Student("student 2", 0002, grades2);

                // add students to list
                List<Student> students = new ArrayList<Student>();
                students.add(student1);
                students.add(student2);

                // define subjects
                Subject subject1 = new Subject("subject 1", 0.5, grades1);
                Subject subject2 = new Subject("subject 2", 0.6, grades2);

                // add subjects to list
                List<Subject> subjects = new ArrayList<Subject>();
                subjects.add(subject1);
                subjects.add(subject2);

                // add students and subjects list to representative
                Representative representative = new Representative();
                representative.setStudents(students);
                representative.setSubjects(subjects);

                // display student list
                representative.displayStudentList();
            }
        }
    \end{minted}
    \texttt{Terminal}
    \begin{minted}[autogobble,breaklines,linenos]{text}
        PS D:\Kuliah>  d:; cd 'd:\Kuliah'; & 'C:\Program Files\Java\jdk-18.0.2.1\bin\java.exe' '-XX:+ShowCodeDetailsInExceptionMessages' '-cp' 'C:\Users\G4CE-PC\AppData\Roaming\Code\User\workspaceStorage\ 80d97a47d24665dc0bce7ab1e048ecbd\redhat.java\jdt_ws\ Kuliah_28156aa7\bin' 'Main' 
        Student and Subject List
        Student  : student 1 
        NIM    : 1.00 
        GPA    : 85.67
        Subject  : subject 1
        Credit : 0.50
        Weight : 85.67
        Student  : student 2
        NIM    : 2.00
        GPA    : 85.67
        Subject  : subject 2
        Credit : 0.60
        Weight : 85.67
    \end{minted}
\end{enumerate}

\end{document}