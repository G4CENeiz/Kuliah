\documentclass[12pt,titlepage]{article}
\usepackage[margin=1.25in]{geometry}
\usepackage{graphicx,amsmath,blindtext,minted}

%% Variables definition
\newcommand{\vSubject}{Object Oriented Programming}
\newcommand{\vSubtitle}{Overloading and Overriding}
\newcommand{\vName}{Muhammad Baihaqi Aulia Asy'ari}
\newcommand{\vNIM}{2241720145}
\newcommand{\vClass}{2I}
\newcommand{\vDepartment}{Information Technology}
\newcommand{\vStudyProgram}{D4 Informatics Engineering}

%% [START] Tikz related stuff
\usepackage{tikz}
\usetikzlibrary{svg.path,calc,shapes.geometric,shapes.misc}
\tikzstyle{terminator} = [rectangle, draw, text centered, rounded corners = 1em, minimum height=2em]
\tikzstyle{preparation} = [chamfered rectangle, chamfered rectangle sep=0.75em, draw, text centered, minimum height = 2em]
\tikzstyle{process} = [rectangle, draw, text centered, minimum height=2em]
\tikzstyle{decision} = [diamond, aspect=2, draw, text centered, minimum height=2em]
\tikzstyle{data}=[trapezium, draw, text centered, trapezium left angle=60, trapezium right angle=120, minimum height=2em]
\tikzstyle{connector} = [line width=0.25mm,->]
%% [END] Tikz related stuff

%% [START] Fancy header related stuff
\usepackage{fancyhdr}
\pagestyle{fancy}
\setlength{\headheight}{15pt} % compensate fancyhdr style
\fancyhead{}
\fancyfoot{}
\fancyfoot[L]{\thepage}
\fancyfoot[R]{\textit{\vSubject - \vSubtitle}}
\renewcommand{\footrulewidth}{0.4pt}% default is 0pt, overline for footer
%% [END] Fancy header related stuff

%% [START] Custom tabular command related stuff
\usepackage{tabularx}
\newcommand{\details}[2]{
    #1 & #2  \\
}
%% [END] Custom tabular command related stuff

%% [START] Figure related stuff
\newcommand{\image}[3][1]{
    \begin{figure}[h]
        \centering
        \includegraphics[#1]{#2}
        \caption{#3}
        \label{#3}
    \end{figure}
}
%% [END] Figure related stuff

%%
\usepackage{pgf-umlcd}

\renewcommand{\umldrawcolor}{black}
\renewcommand{\umlfillcolor}{white}
%%

%% [BEGIN] Custom enumerator
\usepackage{enumitem}
%% [END] Custom enumerator

%% [BEGIN] Paragraph indent
\usepackage{indentfirst}
%% [END] Paragraph indent

\begin{document}
\begin{titlepage}
    \centering
    \vfill
    {\bfseries\LARGE
        \vSubject\\
        \vskip0.25cm
        \vSubtitle
    }
    \vfill
    \includegraphics[width=6cm]{images/polinema-logo.png}
    \vfill
    {
        \textbf{Name}\\
        \vName\\
        \vskip0.5cm
        \textbf{NIM}\\
        \vNIM\\
        \vskip0.5cm
        \textbf{Class}\\
        \vClass\\
        \vskip0.5cm
        \textbf{Department}\\
        \vDepartment\\
        \vskip0.5cm
        \textbf{Study Program}\\
        \vStudyProgram
    }
\end{titlepage}

\newpage

\section{Experiment 1}

\texttt{Main.java}
\begin{minted}[autogobble,breaklines,linenos]{java}
    package experiment1;

    public class Main {
        public static void main(String[] args) {
            Manager[] managers = new Manager[2];
            Staff[] managerOneStaffs = new Staff[2];
            Staff[] managerTwoStaffs = new Staff[3];

            // Managers Instatiation

            managers[0] = new Manager();
            managers[0].setName("Alpha");
            managers[0].setNip("230001");
            managers[0].setBracket("1");
            managers[0].setBonus(5_000_000);
            managers[0].setSection("Alpha Squad");

            managers[1] = new Manager();
            managers[1].setName("Delta");
            managers[1].setNip("230004");
            managers[1].setBracket("1");
            managers[1].setBonus(2_500_000);
            managers[1].setSection("Delta Squad");

            // Alpha Squad Member

            managerOneStaffs[0] = new Staff();
            managerOneStaffs[0].setName("Bravo");
            managerOneStaffs[0].setNip("230002");
            managerOneStaffs[0].setBracket("2");
            managerOneStaffs[0].setOvertime(10);
            managerOneStaffs[0].setOvertimePay(10_000);

            managerOneStaffs[1] = new Staff();
            managerOneStaffs[1].setName("Charlie");
            managerOneStaffs[1].setNip("230003");
            managerOneStaffs[1].setBracket("2");
            managerOneStaffs[1].setOvertime(10);
            managerOneStaffs[1].setOvertimePay(55_000);
            
            managers[0].setStaffs(managerOneStaffs);
            
            // Delta Squad Member
            
            managerTwoStaffs[0] = new Staff();
            managerTwoStaffs[0].setName("Echo");
            managerTwoStaffs[0].setNip("230005");
            managerTwoStaffs[0].setBracket("3");
            managerTwoStaffs[0].setOvertime(15);
            managerTwoStaffs[0].setOvertimePay(5_500);

            managerTwoStaffs[1] = new Staff();
            managerTwoStaffs[1].setName("Foxtrot");
            managerTwoStaffs[1].setNip("230006");
            managerTwoStaffs[1].setBracket("4");
            managerTwoStaffs[1].setOvertime(5);
            managerTwoStaffs[1].setOvertimePay(100_000);

            managerTwoStaffs[2] = new Staff();
            managerTwoStaffs[2].setName("Golf");
            managerTwoStaffs[2].setNip("230007");
            managerTwoStaffs[2].setBracket("3");
            managerTwoStaffs[2].setOvertime(6);
            managerTwoStaffs[2].setOvertimePay(20_000);

            managers[1].setStaffs(managerTwoStaffs);

            // Managers Info

            managers[0].showInfo();
            managers[1].showInfo();
        }
    }

    class Employee {
        private String name;
        private String nip;
        private String bracket;
        private double salary;

        public void setName(String name) {
            this.name = name;
        }

        public void setNip(String nip) {
            this.nip = nip;
        }

        public void setBracket(String bracket) {
            this.bracket = bracket;

            switch (bracket.charAt(0)) {
                case '1': 
                    this.salary = 5_000_000;
                    break;
                case '2': 
                    this.salary = 3_000_000;
                    break;
                case '3':
                    this.salary = 2_000_000;
                    break;
                case '4':
                    this.salary = 1_000_000;
                    break;
                case '5':
                    this.salary = 750_000;
                    break;
            }
        }

        public void setSalary(double salary) {
            this.salary = salary;
        }

        public String getName() {
            return name;
        }

        public String getNip() {
            return nip;
        }

        public String getBracket() {
            return bracket;
        }

        public double getSalary() {
            return salary;
        }
    }

    class Staff extends Employee {
        private int overtime;
        private double overtimePay;

        public void setOvertime(int overtime) {
            this.overtime = overtime;
        }

        public int getOvertime() {
            return overtime;
        }

        public void setOvertimePay(double overtimePay) {
            this.overtimePay = overtimePay;
        }

        public double getOvertimePay() {
            return overtimePay;
        }

        public double getSalary(int overtime, double overtimePay) {
            return super.getSalary() + overtime * overtimePay;
        }

        @Override
        public double getSalary() {
            return super.getSalary() + overtime * overtimePay;
        }

        public void showInfo() {
            System.out.printf("NIP          : %s%n", super.getNip());
            System.out.printf("Name         : %s%n", super.getName());
            System.out.printf("Bracket      : %s%n", super.getBracket());
            System.out.printf("Overtime     : %,d%n", this.getOvertime());
            System.out.printf("Overtime Pay : %,.0f%n", this.getOvertimePay());
            System.out.printf("Salary       : %,.0f%n", this.getSalary());
        }
    }

    class Manager extends Employee {
        private double bonus;
        private String section;
        private Staff[] staffs;

        public void setBonus(double bonus) {
            this.bonus = bonus;
        }

        public double getBonus() {
            return bonus;
        }

        public void setSection(String section) {
            this.section = section;
        }

        public String getSection() {
            return section;
        }

        public void setStaffs(Staff[] staffs) {
            this.staffs = staffs;
        }

        public Staff[] getStaffs() {
            return staffs;
        }

        public void viewStaff() {
            System.out.println("--------------------------------");
            for (Staff staff : staffs) {
                staff.showInfo();
                System.out.println("--------------------------------");
            }
        }

        public void showInfo() {
            System.out.printf("Manager of   : %s%n", this.getSection());
            System.out.printf("NIP          : %s%n", this.getNip());
            System.out.printf("Name         : %s%n", this.getName());
            System.out.printf("Bracket      : %s%n", this.getBracket());
            System.out.printf("Bonus        : %,.0f%n", this.getBonus());
            System.out.printf("Salary       : %,.0f%n", this.getSalary());
            System.out.printf("Section      : %s%n", this.getSection());
            this.viewStaff();
        }

        @Override
        public double getSalary() {
            return super.getSalary() + bonus;
        }
    }
\end{minted}

\texttt{Terminal}
\begin{minted}[autogobble,breaklines,linenos]{text}
    D:\Kuliah\Smt 3\Object Oriented Programming\Week 10\Polymorphism\src\experiment1>java Main.java
    Manager of   : Alpha Squad
    NIP          : 230001
    Name         : Alpha
    Bracket      : 1
    Bonus        : 5,000,000
    Salary       : 10,000,000
    Section      : Alpha Squad
    --------------------------------
    NIP          : 230002
    Name         : Bravo
    Bracket      : 2
    Overtime     : 10
    Overtime Pay : 10,000
    Salary       : 3,100,000
    --------------------------------
    NIP          : 230003
    Name         : Charlie
    Bracket      : 2
    Overtime     : 10
    Overtime Pay : 55,000
    Salary       : 3,550,000
    --------------------------------
    Manager of   : Delta Squad
    NIP          : 230004
    Name         : Delta
    Bracket      : 1
    Bonus        : 2,500,000
    Salary       : 7,500,000
    Section      : Delta Squad
    --------------------------------
    NIP          : 230005
    Name         : Echo
    Bracket      : 3
    Overtime     : 15
    Overtime Pay : 5,500
    Salary       : 2,082,500
    --------------------------------
    NIP          : 230006
    Name         : Foxtrot
    Bracket      : 4
    Overtime     : 5
    Overtime Pay : 100,000
    Salary       : 1,500,000
    --------------------------------
    NIP          : 230007
    Name         : Golf
    Bracket      : 3
    Overtime     : 6
    Overtime Pay : 20,000
    Salary       : 2,120,000
    --------------------------------
\end{minted}

\newpage

\section{Exercise}
\begin{minted}[autogobble,breaklines,linenos]{java}
    public class PerkalianKu {
        void perkalian(int a, int b) {
            System.out.println(a * b);
        }

        void perkalian(int a, int b, int c) {
            System.out.println(a * b * c);
        }

        public static void main(String[] args) {
            PerkalianKu objek = new PerkalianKu();

            objek.perkalian(25, 43);
            objek.perkalian(34, 23, 56);
        }
    }
\end{minted}
\subsection{\textnormal{Dari source coding diatas terletak dimanakah overloading?}}
\texttt{Answer:}\\
line 2 and line 6 declared the method overloading.
\subsection{\textnormal{Jika terdapat overloading ada berapa jumlah parameter yang berbeda?}}
\texttt{Answer:}\\
both method have a diferent amount of parameter with the first having 2 and the second having 3.

\newpage

\begin{minted}[autogobble,breaklines,linenos]{java}
    public class PerkalianKu {
        void perkalian(int a, int b) {
            System.out.println(a * b);
        }

        void perkalian(double a, double b) {
            System.out.println(a * b);
        }

        public static void main(String[] args) {
            PerkalianKu objek = new PerkalianKu();

            objek.perkalian(25, 43);
            objek.perkalian(34.56, 23.7);
        }
    }
\end{minted}
\subsection{\textnormal{Dari source coding diatas terletak dimanakah overloading?}}
\texttt{Answer:}\\
line 2 and line 6 declared the method overloading.
\subsection{\textnormal{Jika terdapat overloading ada berapa tipe parameter yang berbeda?}}
\texttt{Answer:}\\
it is a diferent data type for each parameter.

\newpage

\begin{minted}[autogobble,breaklines,linenos]{java}
    class Ikan {
        public void swim() {
            System.out.println("Ikan bisa berenang");
        }
    }
    class Piranha extends Ikan {
        public void swim() {
            System.out.println("Piranha bisa makan daging");
        }
    }
    public class Fish {
        Ikan a = new Ikan();
        Ikan b = new Piranha();
        a.swim();
        b.swim();
    }
\end{minted}
\subsection{\textnormal{Dari source coding diatas terletak dimanakah overriding?}}
\texttt{Answer:}\\
line 7 override the method from its superclass in line 2.
\subsection{\textnormal{Jabarkanlah apabila sourcoding diatas jika terdapat overriding?}}
\texttt{Answer:}\\
yes there is a method override in the source code. The extended Piranha class override the swim method it inherited from the Ikan superclass.

\newpage

\section{Assignment}
\subsection{Overloading}
\begin{minted}[autogobble,breaklines,linenos]{java}
    package assignment;

    public class Triangle {
        private int angle;

        public int sumAngle(int angleA) {
            return 180 - angleA;
        }

        public int sumAngle(int angleA, int angleB) {
            return 180 - angleA - angleB;
        }

        public int circumference(int sideA, int sideB, int sideC) {
            return sideA + sideB + sideC;
        }

        public double circumference(int sideA, int sideB) {
            return sideA + sideB + Math.sqrt((sideA * sideA) + (sideB * sideB));
        }

        public void setAngle(int angle) {
            this.angle = angle;
        }

        public int getAngle() {
            return angle;
        }
    }
\end{minted}

\newpage

\subsection{Overriding}
\begin{minted}[autogobble,breaklines,linenos]{java}
    package assignment;

    public class Human {
        public void breath() {
            System.out.println("breath");
        }

        public void eat() {
            System.out.println("Om Nom Nom");
        }
    }

    class Lecturer extends Human {
        @Override
        public void eat() {
            System.out.println("Slurp");
        }

        public void overtime() {
            System.out.println("Doing overtime just like a horse");
        }
    }

    class Student extends Human {
        @Override
        public void eat() {
            System.out.println("Chomp Chomp Chomp");
        }

        public void sleep() {
            System.out.println("ZZZ - Atarashii Gakko No Leaders");
        }
    }
\end{minted}

\end{document}