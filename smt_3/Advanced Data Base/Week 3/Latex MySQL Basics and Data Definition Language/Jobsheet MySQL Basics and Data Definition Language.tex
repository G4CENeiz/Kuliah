\documentclass[12pt,titlepage]{article}
\usepackage[margin=1.25in]{geometry}
\usepackage{graphicx,amsmath,blindtext,minted}

%% Variables definition
\newcommand{\vSubject}{Advance Data Base}
\newcommand{\vSubtitle}{MySQL Basics and Data Definition Language}
\newcommand{\vName}{Muhammad Baihaqi Aulia Asy'ari}
\newcommand{\vNIM}{2241720145}
\newcommand{\vClass}{2I}
\newcommand{\vDepartment}{Information Technology}
\newcommand{\vStudyProgram}{D4 Informatics Engineering}

%% [START] Tikz related stuff
\usepackage{tikz}
\usetikzlibrary{svg.path,calc,shapes.geometric,shapes.misc}
\tikzstyle{terminator} = [rectangle, draw, text centered, rounded corners = 1em, minimum height=2em]
\tikzstyle{preparation} = [chamfered rectangle, chamfered rectangle sep=0.75em, draw, text centered, minimum height = 2em]
\tikzstyle{process} = [rectangle, draw, text centered, minimum height=2em]
\tikzstyle{decision} = [diamond, aspect=2, draw, text centered, minimum height=2em]
\tikzstyle{data}=[trapezium, draw, text centered, trapezium left angle=60, trapezium right angle=120, minimum height=2em]
\tikzstyle{connector} = [line width=0.25mm,->]
%% [END] Tikz related stuff

%% [START] Fancy header related stuff
\usepackage{fancyhdr}
\pagestyle{fancy}
\setlength{\headheight}{15pt} % compensate fancyhdr style
\fancyhead{}
\fancyfoot{}
\fancyfoot[L]{\thepage}
\fancyfoot[R]{\textit{\vSubject - \vSubtitle}}
\renewcommand{\footrulewidth}{0.4pt}% default is 0pt, overline for footer
%% [END] Fancy header related stuff

%% [START] Custom tabular command related stuff
\usepackage{tabularx}
\newcommand{\details}[2]{
    #1 & #2  \\
}
%% [END] Custom tabular command related stuff

%% [START] Figure related stuff
\newcommand{\image}[3][1]{
    \begin{figure}[h]
        \centering
        \includegraphics[#1]{#2}
        \caption{#3}
        \label{#3}
    \end{figure}
}
%% [END] Figure related stuff

%%
\usepackage{pgf-umlcd}

\renewcommand{\umldrawcolor}{black}
\renewcommand{\umlfillcolor}{white}
%%

%% [BEGIN] Custom enumerator
\usepackage{enumitem}
%% [END] Custom enumerator

%% [BEGIN] Paragraph indent
\usepackage{indentfirst}
%% [END] Paragraph indent

\begin{document}
\begin{titlepage}
    \centering
    \vfill
    {\bfseries\LARGE
        \vSubject\\
        \vskip0.25cm
        \vSubtitle
    }
    \vfill
    \includegraphics[width=6cm]{images/polinema-logo.png}
    \vfill
    {
        \textbf{Name}\\
        \vName\\
        \vskip0.5cm
        \textbf{NIM}\\
        \vNIM\\
        \vskip0.5cm
        \textbf{Class}\\
        \vClass\\
        \vskip0.5cm
        \textbf{Department}\\
        \vDepartment\\
        \vskip0.5cm
        \textbf{Study Program}\\
        \vStudyProgram
    }
\end{titlepage}

\newpage

\section*{Practicum}

\begin{enumerate}
    \item Buka prompt jalankan perintah berikut ini :\\
    C:$\backslash>$Program Files$\backslash$xampp$\backslash$mysql$\backslash$bin$>$mysql -u root -p (enter) \\
    \includegraphics[width=0.9\textwidth]{images/figures/practicum-1.PNG}
    \item Buatlah sebuah database dengan nama db\textunderscore polinema \\
    \includegraphics[width=0.9\textwidth]{images/figures/practicum-2.PNG}
    \item Buatlah beberapa tabel dalam database tersebut sesuai dengan kriteria berikut: \\
    Tabel \texttt{prodi} \\
    \resizebox{.9\textwidth}{!}{
        \begin{tabular}{|p{0.2\textwidth}|p{0.6\textwidth}|}
            \hline
            \textbf{Field} & \textbf{Type Data} \\
            \hline
            kode\textunderscore prodi & VARCHAR (6) PRIMARY KEY \\
            \hline
            nama\textunderscore prodi & VARCHAR (30) \\
            \hline
        \end{tabular}
    } \\
    \includegraphics[width=.9\textwidth]{images/figures/practicum-4.PNG}
    \newpage
    \item Tabel \texttt{mahasiswa} \\
    \resizebox{.9\textwidth}{!}{
        \begin{tabular}{|p{0.2\textwidth}|p{0.6\textwidth}|}
            \hline
            \textbf{Field} & \textbf{Type Data} \\
            \hline
            nim & INT (8) PRIMARY KEY \\
            \hline
            nama\textunderscore mhs & VARCHAR (50) \\
            \hline
            jenis\textunderscore kelamin & ENUM ('L','P') DEFAULT 'L' \\
            \hline
            alamat & VARCHAR (50) \\
            \hline
            kota & VARCHAR (20) DEFAULT 'MALANG' \\
            \hline
            asal\textunderscore sma & VARCHAR (30) \\
            \hline
            no\textunderscore hp & VARCHAR (12) \\
            \hline
            umur & INT \\
            \hline
            kode\textunderscore prodi & VARCHAR (6) FOREIGN KEY fk0 (kode\textunderscore prodi) REFERENSCES prodi (kode\textunderscore prodi) \\
            \hline
        \end{tabular}
    } \\
    \includegraphics[width=.9\textwidth]{images/figures/practicum-3.PNG}
    \item Tabel \texttt{mata\textunderscore kuliah} \\
    \resizebox{.9\textwidth}{!}{
        \begin{tabular}{|p{0.2\textwidth}|p{0.6\textwidth}|}
            \hline
            \textbf{Field} & \textbf{Type Data} \\
            \hline
            mk\textunderscore id & VARCHAR (10) PRIMARY KEY \\
            \hline
            nama\textunderscore mk & VARCHAR (50) \\
            \hline
            jumlah\textunderscore jam & FLOAT (4,2) \\
            \hline
            sks & INTEGER \\
            \hline
        \end{tabular}
    } \\
    \includegraphics[width=.9\textwidth]{images/figures/practicum-5.PNG}
    \newpage
    \item Tabel \texttt{ruang} \\
    \resizebox{.9\textwidth}{!}{
        \begin{tabular}{|p{0.2\textwidth}|p{0.6\textwidth}|}
            \hline
            \textbf{Field} & \textbf{Type Data} \\
            \hline
            ruang\textunderscore id & VARCHAR (10) PRIMARY KEY \\
            \hline
            nama\textunderscore ruang & VARCHAR (50) \\
            \hline
            kapasitas & INTEGER \\
            \hline
        \end{tabular}
    } \\
    \includegraphics[width=.9\textwidth]{images/figures/practicum-6.PNG}
    \item Tabel \texttt{dosen} \\
    \resizebox{.9\textwidth}{!}{
        \begin{tabular}{|p{0.2\textwidth}|p{0.6\textwidth}|}
            \hline
            \textbf{Field} & \textbf{Type Data} \\
            \hline
            nidn & INTEGER (20) PRIMARY KEY \\
            \hline
            nama\textunderscore dosen & VARCHAR (50) \\
            \hline
            status & ENUM ('PNS','KONTRAK') DEFAULT 'PNS' \\
            \hline
            jenis\textunderscore kelamin & ENUM ('L','P') DEFAULT 'L' \\
            \hline
            no\textunderscore hp & VARCHAR (15) \\
            \hline
        \end{tabular}
    } \\
    \includegraphics[width=.9\textwidth]{images/figures/practicum-7.PNG}
    \newpage
    \item \textbf{$<$Soal$>$} \\ Tambahkan sebuah kolom agama (varchar(10)) pada tabel mahasiswa sebagai kolom terakhir \\
    \textcolor{red}{Catat : Buat Screenshot dari perintah yang anda ketikkan} \\
    \includegraphics[width=.9\textwidth]{images/figures/practicum-8.PNG}
    \newpage
    \item \textbf{$<$Soal$>$} \\ Tambahkan kolom alamat(varchar(50)) pada tabel dosen sebagai kolom terakhir \\
    \textcolor{red}{Catat : Buat Screenshot dari perintah yang anda ketikkan} \\
    \includegraphics[width=.9\textwidth]{images/figures/practicum-9.PNG}
    \newpage
    \item \textbf{$<$Soal$>$} \\ Lakukan insert data ke dalam tabel-tabel yang ada pada pada database \newline db\textunderscore polinema sesuai dengan field, tipe data dan panjang datanya \\
    \textcolor{red}{Catat : Buat Screenshot dari perintah yang anda ketikkan} \\
    \includegraphics[width=.9\textwidth]{images/figures/practicum-10.PNG}
    \newpage
    \item \textbf{$<$Soal$>$} \\ Tampilkan semua tabel yang ada didalam database db\textunderscore polinema \\
    \textcolor{red}{Catat : Buat Screenshot dari perintah yang anda ketikkan} \\
    \includegraphics[width=.9\textwidth]{images/figures/practicum-11.PNG}\newpage
    \item \textbf{$<$Soal$>$} \\ Tampilkan semua isi tabel yang ada didalam tabel mahasiswa \\ 
    \textcolor{red}{Catat : Buat Screenshot dari perintah yang anda ketikkan} \\
    \includegraphics[width=.9\textwidth]{images/figures/practicum-12.PNG}
    \item \textbf{$<$Soal$>$} \\ Tampilkan struktur(metadata) tabel mahasiswa 
    \textcolor{red}{Catat : Buat Screenshot dari perintah yang anda ketikkan} \\
    \includegraphics[width=.9\textwidth]{images/figures/practicum-13.PNG}
    \newpage
    \item \textbf{$<$Soal$>$} \\ hilangkan kolom asal\textunderscore sma yang terdapat didalam tabel mahasiswa \\ 
    \textcolor{red}{Catat : Buat Screenshot dari perintah yang anda ketikkan} \\
    \includegraphics[width=.9\textwidth]{images/figures/practicum-14.PNG}
\end{enumerate}

\newpage
\section*{Tugas}
\begin{enumerate}
    \item \textbf{Buatlah basis data Akademik dengan data sebagai berikut :} \\
    \resizebox{.9\textwidth}{!}{
        \begin{tabular}{|l|l|l|l|l|l|l|l|}
            \hline
            No\textunderscore Mhs & Nama\textunderscore mhs & Jurusan & Kd\textunderscore MK & Nama\textunderscore mk & Kd\textunderscore Dosen & Nm\textunderscore Dosen & nilai \\
            \hline
            1921001 & Aminah & MI & MI350 & Basis Data & B104 & Ati & 85 \\
            \hline
            1921001 & Budiman & MI & MI465 & Pemrograman & B105 & Dita & 87 \\
            \hline
            1921002 & Carina & MI & MI465 & Pemrograman & B105 & Dita & 85 \\
            \hline
            1921003 & Della & TI & TI201 & Mobile & C102 & Leo & 78 \\
            \hline
            1921004 & Firda & TI & TI201 & Mobile & C102 & Leo & 80 \\
            \hline
        \end{tabular}
    } \\ 
    \includegraphics[width=.9\textwidth]{images/figures/tugas-1_preamble.PNG}
    \begin{enumerate}
        \item deskripsikan struktur data dari table-tabel berikut serta isikan datanya: \\
        Tabel Mahasiswa \{No\textunderscore Mhs, Nama\textunderscore mhs\} \\
        Tabel Mata\textunderscore Kuliah \{Kd\textunderscore MK, Nama\textunderscore MK\} \\
        Tabel nilai \{No\textunderscore Mhs, Kode\textunderscore MK\} \\
        tambahkan kolom Jurusan pada tabel Mahasiswa di kolom terakhir \\
        \includegraphics[width=.85\textwidth]{images/figures/tugas-1_a_1.PNG} \\
        \includegraphics[width=.85\textwidth]{images/figures/tugas-1_a_2.PNG}
        \item tambahkan kolom Kode Dosen pada tabel Mata\textunderscore Kuliah \\
        \includegraphics[width=.85\textwidth]{images/figures/tugas-1_b.PNG}
        \newpage
        \item tambahkan kolom nilai pada tabel nilai serta berikanlah kunci foreign key \\
        \includegraphics[width=.85\textwidth]{images/figures/tugas-1_c.PNG}
        \item tambahkan Tabel Dosen dengan atributnya Kd\textunderscore Dosen dan Nama Dosen \\
        \includegraphics[width=.85\textwidth]{images/figures/tugas-1_d.PNG}
        \newpage
        \item tampilkan semua data yang ada pada tiap tabel \\
        \includegraphics[width=.85\textwidth]{images/figures/tugas-1_e_1.PNG} \\
        \includegraphics[width=.85\textwidth]{images/figures/tugas-1_e_2.PNG}
    \end{enumerate}
    \item \textbf{Buatlah basis data Pegawai yang terdiri dari tabel sebagai berikut :}
    \resizebox{.9\textwidth}{!}{
        \begin{tabular}{|l|l|l|l|l|l|}
            \hline
            Noproyek & NamaProyek & Nopegawai & NamaPegawai & Golongan & BesarGaji \\
            \hline
            NP001 & BRR & Peg01 & Anton & A & 1.000.000 \\
            \hline
            NP001 & BRR & Peg02 & Paula & B & 900.000 \\
            \hline
            NP001 & BRR & Peg06 & Koko & C & 750.000 \\
            \hline
            NP002 & PEMDA & Peg01 & Anton & A & 1.000.000 \\
            \hline
            NP002 & PEMDA & Peg12 & Sita & B & 900.000 \\
            \hline
            NP002 & PEMDA & Peg14 & Yusni & B & 900.000 \\
            \hline
            NP003 & CBR & Peg02 & Paula & B & 900.000 \\
            \hline
            NP003 & CBR & Peg03 & Daniar & C & 750.000 \\
            \hline
            NP003 & CBR & Peg04 & Lubis & C & 750.000 \\
            \hline
            NP004 & ASK & Peg07 & Keni & B & 900.000 \\
            \hline
            NP004 & ASK & Peg08 & Sofi & B & 900.000 \\
            \hline
            NP004 & ASK & Peg06 & Yuni & C & 750.000 \\
            \hline
            NP005 & OB & Peg15 & Udin & D & 500.000 \\
            \hline
            NP005 & OB & Peg16 & Didit & D & 500.000 \\
            \hline
            NP005 & OB & Peg17 & Dani & D & 500.000 \\
            \hline
        \end{tabular}
    }\\ 
    \includegraphics[width=.9\textwidth]{images/figures/tugas-2_preamble.PNG}
    \begin{enumerate}
        \item Deskripsikan struktur data dari table-tabel berikut serta isikan datanya: \\
        Table Pegawai \{Nopegawai, NamaPegawai\} \\
        Tabel Golongan \{Golongan\} \\
        Tabel Proyek \{Noproyek\} \\
        Tabel Proyekpegawai \{Noproyek\} \\
        \includegraphics[width=.85\textwidth]{images/figures/tugas-2_a_1.png} \\
        \includegraphics[width=.85\textwidth]{images/figures/tugas-2_a_2.png}
        \item Tambahkan kolom Golongan pada tabel Pegawai di kolom terakhir \\
        \includegraphics[width=.85\textwidth]{images/figures/tugas-2_b.png}
        \newpage
        \item Tambahkan kolom BesarGaji pada tabel Golongan di kolom terakhir \\
        \includegraphics[width=.85\textwidth]{images/figures/tugas-2_c.png}
        \item Tambahkan kolom NamaProyek pada table Proyek \\
        \includegraphics[width=.85\textwidth]{images/figures/tugas-2_d.png}
        \item Tambahkan kolom NoPegawai pada table Proyekpegawai serta berikanlah kunci foreign key \\
        \includegraphics[width=.85\textwidth]{images/figures/tugas-2_e.png}
        \newpage
        \item Tampilkan semua data yang ada pada tiap tabel \\
        \includegraphics[width=.42\textwidth]{images/figures/tugas-2_f_1.png}
        \includegraphics[width=.42\textwidth]{images/figures/tugas-2_f_2.png}  \\
        \includegraphics[width=.85\textwidth]{images/figures/tugas-2_f_3.png}
    \end{enumerate}
\end{enumerate}

\end{document}