\documentclass[12pt,titlepage]{article}
\usepackage[margin=1.25in]{geometry}
\usepackage{graphicx,amsmath,blindtext,minted}

%% Variables definition
\newcommand{\vSubject}{Data Structure and Algorithm Practicum}
\newcommand{\vSubtitle}{Basic Programming}
\newcommand{\vName}{Muhammad Baihaqi Aulia Asy'ari}
\newcommand{\vNIM}{2241720145}
\newcommand{\vClass}{1I}
\newcommand{\vDepartment}{Information Technology}
\newcommand{\vStudyProgram}{D4 Informatics Engineering}

%% [START] Tikz related stuff
\usepackage{tikz}
\usetikzlibrary{svg.path,calc,shapes.geometric,shapes.misc}
\tikzstyle{terminator} = [rectangle, draw, text centered, rounded corners = 1em, minimum height=2em]
\tikzstyle{preparation} = [chamfered rectangle, chamfered rectangle sep=0.75em, draw, text centered, minimum height = 2em]
\tikzstyle{process} = [rectangle, draw, text centered, minimum height=2em]
\tikzstyle{decision} = [diamond, aspect=2, draw, text centered, minimum height=2em]
\tikzstyle{data}=[trapezium, draw, text centered, trapezium left angle=60, trapezium right angle=120, minimum height=2em]
\tikzstyle{connector} = [line width=0.25mm,->]
%% [END] Tikz related stuff

%% [START] Fancy header related stuff
\usepackage{fancyhdr}
\pagestyle{fancy}
\setlength{\headheight}{15pt} % compensate fancyhdr style
\fancyhead{}
\fancyfoot{}
\fancyfoot[L]{\thepage}
\fancyfoot[R]{\textit{\vSubject - \vSubtitle}}
\renewcommand{\footrulewidth}{0.4pt}% default is 0pt, overline for footer
%% [END] Fancy header related stuff

%% [START] Custom tabular command related stuff
\usepackage{tabularx}
\newcommand{\details}[2]{
    #1 & #2  \\
}
%% [END] Custom tabular command related stuff

%% [START] Figure related stuff
\newcommand{\image}[3][1]{
    \begin{figure}[h]
        \centering
        \includegraphics[#1]{#2}
        \caption{#3}
        \label{#3}
    \end{figure}
}
%% [END] Figure related stuff

%%
\usepackage{pgf-umlcd}

\renewcommand{\umldrawcolor}{black}
\renewcommand{\umlfillcolor}{white}
%%

%%
\usepackage{pdfpages}
%%

\begin{document}
\begin{titlepage}
    \centering
    \vfill
    {\bfseries\LARGE
        \vSubject\\
        \vskip0.25cm
        \vSubtitle
    }
    \vfill
    \includegraphics[width=6cm]{images/polinema-logo.png}
    \vfill
    {
        \textbf{Name}\\
        \vName\\
        \vskip0.5cm
        \textbf{NIM}\\
        \vNIM\\
        \vskip0.5cm
        \textbf{Class}\\
        \vClass\\
        \vskip0.5cm
        \textbf{Department}\\
        \vDepartment\\
        \vskip0.5cm
        \textbf{Study Program}\\
        \vStudyProgram
    }
\end{titlepage}

\newpage

\setcounter{section}{1}
\setcounter{subsection}{1}
\subsection{Conditional Statements}
\subsubsection{Practicum of Conditional Statements}
\texttt{Questions}

\begin{enumerate}
    \item Create a program to calculate final score of students with these compositions. 20\% of final score comes from assignment score, 35\% from midterm score, and 45\% from final exam. Each input score ranges from 0 – 100. Once the final score is determined, do the conversion as follows: \mbox{}\\
    \begin{tabular}{|c|c|}
        \hline
        \textbf{Score} & \textbf{Alphabet} \\
        \hline
        80 $<$ N $\le$ 100 & A \\
        \hline
        73 $<$ N $\le$ 80 & B+ \\
        \hline
        65 $<$ N $\le$ 73 & B \\
        \hline
        60 $<$ N $\le$ 65 & C+ \\
        \hline
        50 $<$ N $\le$ 60 & C \\
        \hline
        39 $<$ N $\le$ 50 & D \\
        \hline
        N $\le$ 39 & E \\
        \hline
    \end{tabular}
    \mbox{}\\
    If the acquired alphabets are A, B+, B, C+, C then the student is \textbf{passed}. Otherwise, the student is \textbf{failed}.
    \begin{itemize}
        \item The program needs inputs for assignment score, midterm, final exam score
        \item The output will be the final score, its alphabet, and information whether they \textbf{passed} or \textbf{failed}
    \end{itemize}
    \texttt{Example:}
    \begin{minted}[autogobble,breaklines]{text}
        Program Menghitung Nilai Akhir
        ======================
        Masukkan Nilai Tugas: 85
        Masukkan Nilai UTS: 60
        Masukkan Nilai UAS: 82
        ======================
        ======================
        nilai akhir : 74.9
        Nilai Huruf :B+
        ======================
        SELAMAT LULUS
    \end{minted}
\end{enumerate}

\newpage

\begin{minted}[autogobble,breaklines]{java}
    import java.util.Scanner;

    public class ConditonalStatement {
        final static Scanner input = new Scanner(System.in);
        static double AssignmentPercentage = 0.2;
        static double MidtermPercentage = 0.35;
        static double finalExamPercentage = 0.45;
        public static void main(String[] args) {
            System.out.println("Program Menghitung Nilai Akhir");
            bar();
            int assignmentScore =   PromptInput("assignment");
            int midtermScore    =   PromptInput("midterm");
            int finalExamScore  =   PromptInput("finals exam");
            bar();
            double finalScore = (assignmentScore * AssignmentPercentage) + (midtermScore * MidtermPercentage) + (finalExamScore * finalExamPercentage);
            String scorePredicate = scorePredicate(finalScore);
            String status = (finalExamScore > 50) ? "Passed" : "Failed"; 
            bar();
            System.out.printf("Final score  : %.1f\n", finalScore);
            System.out.printf("Predicate    : %s\n", scorePredicate);
            bar();
            System.out.println(status);
        }
        static void bar() {
            System.out.println("================================");
        }
        static int PromptInput(String prompt) {
            System.out.printf("Enter %-12s score: ",prompt);
            int userInput = input.nextInt();
            input.nextLine();
            return userInput;
        }
        static String scorePredicate(double finalScore) {
            if (finalScore > 80) {
                return "A";
            } else if (finalScore > 73) {
                return "B+";
            } else if (finalScore > 65) {
                return "B";
            } else if (finalScore > 60) {
                return "C+";
            } else if (finalScore > 50) {
                return "C";
            } else if (finalScore > 39) {
                return "D";
            } else {
                return "E";
            }
        }
    }
\end{minted}

\subsection{Loops}

\subsubsection{Practicum of Loops}

\texttt{Question}

\begin{enumerate}
    \item Create a program that can display the day from Monday to Sunday repetitively with days amount is n, the n will be the last 2 digits from your NIM.\mbox{}\\
    *if n $<$ 10, then add 10 (n+=10)\mbox{}\\
    Example: \mbox{}\\
    Input NIM: 2041720010, then n = 10\mbox{}\\
    \texttt{OUTPUT: Monday Tuesday Wednesday Thursday Friday Saturday Sunday Monday Tuesday Wednesday}\mbox{}\\
    2\textsuperscript{nd} Example: \mbox{}\\
    Input NIM: 2041720002, then n = 12\mbox{}\\
    OUTPUT: Monday Tuesday Wednesday Thursday Friday Saturday Sunday Monday Tuesday Wednesday Thursday Friday \mbox{}\\
    \mbox{}\\
    Example result: \mbox{}\\
    \begin{minted}[autogobble]{text}
        Masukkan NIM :201234501
        =====================
        n : 11
        senin selasa rabu kamis jumat sabtu minggu senin selasa rabu kamis
    \end{minted}
\end{enumerate}

\texttt{Answer: }

\newpage

\begin{minted}[autogobble,breaklines]{java}
    import java.util.Scanner;

    public class Loops {
        final static Scanner input = new Scanner(System.in);
        public static void main(String[] args) {
            String[] days = {"Monday ", "Tuesday ", "Wednesday ", "Thursday ", "Friday ", "Saturday ", "Sunday "};          
            String NIM = getNIM();
            int last2digits = Integer.parseInt(String.format("%c%c", NIM.charAt(NIM.length() - 2), NIM.charAt(NIM.length() - 1)));
            int limit = last2digits < 10 ? last2digits + 10 : last2digits;
            for (int i = 0; i < limit; i++) {
                System.out.print(days[i % days.length]);
            }
        }
        static String getNIM() {
            while (true) {
                System.out.print("Enter NIM: ");
                String userInput = input.next();
                if (validateNIM(userInput)) return userInput;
                System.out.println("Please enter a 10 digit number");
            }
        }
        static boolean validateNIM(String numbers) {
            int length = numbers.length();
            if (length != 10) return false;
            for (int i = 0; i < length; i++) {
                switch (numbers.charAt(i)) {
                    case '1', '2', '3', '4', '5', '6', '7', '8', '9', '0' -> {
                    }
                    default -> {
                        return false;
                    }
                }
            }
            return true;
        }
    }
\end{minted}

\newpage

\subsection{Array}

\subsubsection{Practicum of Array}

\texttt{Question}

\begin{enumerate}
    \item RoyalGarden is a flower shop that has many branches. Every day, the sold flowers and its stock has recorded as follows \mbox{}\\
    \begin{tabular}{|c|c|c|c|c|}
        \hline
        $\backslash$ & Aglaonema & Taro & Alocasia & Rose \\
        \hline
        Royal Garden 1 & 10 & 5 & 15 & 7 \\
        \hline
        Royal Garden 2 & 6 & 11 & 9 & 12 \\
        \hline
        Royal Garden 3 & 2 & 10 & 10 & 5 \\
        \hline
        Royal Garden 4 & 5 & 7 & 12 & 9 \\
        \hline
    \end{tabular} \mbox{}\\
    The price for each Aglaonema is 75.000, Taro is 50.000, Alocasia is 60.000, and Rose is 10.000. Please help RoyalGarden to create a program that can calculate:
    \begin{enumerate}
        \item Stock for each flower through all branches
        \item If there is an additional information about a stock has decreased since the flowers are wither or dead on RoyalGarden 1 branch. Those dead flowers are 1 Aglaonema, 2 Taros, and 5 Roses. Please calculate the income of RoyalGarden 1 if all flowers are sold out.
    \end{enumerate}
\end{enumerate}

\texttt{Answer: }

\begin{minted}[autogobble,breaklines]{java}
    public class Array {
        public static void main(String[] args) {
            int[][] stock = {
                {10, 5, 15, 7},
                {6, 11, 9, 12},
                {2, 10, 10, 5},
                {5, 7, 12, 9}
            };
            String[] flowerName = {"Aglaonema", "Taro", "Alocasia", "Rose"};
            int[] flowerStock = new int[stock[0].length];
            int[] flowerPrice = {75_000, 50_000, 60_000, 10_000};
            int[] withered = {1, 2, 0, 5};

            for (int flowerID = 0; flowerID < stock[0].length; flowerID++) {
                for (int branchID = 0; branchID < stock.length; branchID++) {
                    flowerStock[flowerID] += stock[branchID][flowerID];
                }
            }
            for (int i = 0; i < flowerName.length; i++) {
                System.out.printf("%-10s: %d\n", flowerName[i], flowerStock[i]);
            }
            int branchProfit = 0;
            for (int flowerID = 0; flowerID < flowerName.length; flowerID++) {
                branchProfit += (stock[0][flowerID]-withered[flowerID]) * flowerPrice[flowerID];
            }
            System.out.printf("Branch 1 income: %,d", branchProfit);
        }
    }

\end{minted}

\subsection{Function}

\subsubsection{Practicum of Function}

\texttt{Question}

\begin{enumerate}
    \item Create 2 functions for:
    \begin{enumerate}
        \item Display Fibonacci row using loop
        \item Display Fibonacci row using recursive function
        \mbox{}\\ Notes: 
        \mbox{}\\ Fibonacci row: 0, 1, 1, 2, 3, 5, 8, 13, 21
    \end{enumerate}
\end{enumerate}

\texttt{Answer: }

\begin{minted}[autogobble,breaklines]{java}
    public class function {
        static int n1=0,n2=1,n3=0; 
        static void printFibonacci(int count){    
            if(count>0){    
                n3 = n1 + n2;    
                n1 = n2;    
                n2 = n3;    
                System.out.print(" "+n3);   
                printFibonacci(count-1);    
            }    
        }
        public static void main(String[] args) {
            int secondPrev = 0, prev = 1, cur, limit = 9;
            System.out.printf("%s, %s", secondPrev, prev);
            for (int i = 2; i < limit; ++i) {
                cur = secondPrev + prev;
                System.out.printf(", %s", cur);
                secondPrev = prev;
                prev = cur;
            }
            int count=10;    
            System.out.print(n1+" "+n2);    
            printFibonacci(count-2);
        }
    }
\end{minted}

\subsubsection{Assignment}

\begin{enumerate}
    \item Smile Laundry is a laundry service that costs its customer as follows:
    \begin{enumerate}
        \item item Cost for each 1kg clothes is Rp 4.500
        \item If the customer does laundry more than 10kg clothes, they will get 5\% discount
    \end{enumerate}
    \mbox{}\\ 
    Today, the laundry has 4 customers, those are Ani, Budi, Bina, and Cita. Ani brought 4kg clothes, Budi brought 15kg clothes, Bina brought 6kg, and Cita brought 11kg. Create a program to calculate the income of Smile Laundry at that day.
    \begin{minted}[autogobble,breaklines]{java}
       public class Assignment1 {
            static double income = 0;
            public static void calculatePrice(String name, int laundry) {
                double pricePerkilo = 4_500;
                double discount = 1.00;
                if (laundry > 10) discount = 0.95;
                double cost = laundry * pricePerkilo * discount;
                income += cost; 
                System.out.printf("%s laundries cost: %.0f\n", name, cost);
            }
            public static void main(String[] args) {
                int[] customerLaundries = {4, 15, 6, 11};
                String[] customerName = {"Ani", "Budi", "Bina", "Cita"};
                for (int i = 0; i < customerLaundries.length; i++) {
                    calculatePrice(customerName[i], customerLaundries[i]);
                }
                System.out.printf("Total income: %.0f\n", income);
            }
        } 
    \end{minted}
    \item Somebody saves 1 million rupiahs in a bank. With its interest is 2\% for each month, then in what month does the customer balance reach 1.5 million? Create a program for this case study.
    \begin{minted}[autogobble,breaklines]{java}
        public class Assignment2 {
            public static void main(String[] args) {
                double balance = 1_000_000;
                double target = 1_500_000;
                int month = 0;
                System.out.printf("Initial balance: %,.2f\n", balance);
                while (balance < target) {
                    balance *= 1.02;
                    month++;
                    System.out.printf("%d month balance: %,.2f\n", month, balance);
                }
                System.out.printf("It takes %d month to reach the target of %,.0f\n", month, target);
            }
        }
    \end{minted}
    \item Create a program that can display even numbers from 2 until nth row, unless the even number is a multiple of 4. \mbox{}\\
    Example: \mbox{}\\
    Input of n: 5 \mbox{}\\
    output: 2, 6, 10, 14, 18
    \begin{minted}[autogobble,breaklines]{java}
        import java.util.Scanner;

        public class Assignment3 {
            public static void main(String[] args) {
                Scanner input = new Scanner(System.in);

                System.out.print("Input of n: ");
                int limit = input.nextInt();
                int currentNumber = 2;
                input.close();
                
                System.out.print("output: ");
                for (int i = 0; i < limit; i++) {
                    if (currentNumber % 4 != 0) System.out.printf("%d, ", currentNumber); 
                    else --i; 
                    currentNumber += 2;
                }
            }
        }
    \end{minted}
    \item Create a program that includes a function to:
    \begin{enumerate}
        \item Menu (to choose a calculation for area of triangle / rectangle / circle)
        \item Calculate area of triangle
        \item Calculate area of rectangle
        \item Calculate area of circle
    \end{enumerate}
    \begin{minted}[autogobble,breaklines]{java}
        import java.util.Scanner;

        public class Assignment4 {
            static Scanner input = new Scanner(System.in);
            static void menu() {
                System.out.println("Choose a shape to calculate the area of");
                System.out.println("Type in the number");
                System.out.println("1. Calculate area of triangle");
                System.out.println("2. Calculate area of rectangle");
                System.out.println("3. Calculate area of circle");
                System.out.print("Menu: ");
                String menu = input.next();
                switch (menu) {
                    case "1" -> calculateAreaOfTriangle();
                    case "2" -> calculateAreaOfRectangle();
                    case "3" -> calculateAreaOfCircle();
                }
            }
            static void calculateAreaOfTriangle() {
                System.out.print("Enter length of the side: ");
                double side = input.nextDouble();
                System.out.print("Enter length of the height: ");
                double height = input.nextDouble();


                double area = 0.5 * side * height;
                System.out.printf("The area of triangle with side of %.2f and height of %.2f is %.2f", side, height, area);
            }
            static void calculateAreaOfRectangle() {
                System.out.print("Enter length of the width: ");
                double width = input.nextDouble();
                System.out.print("Enter length of the length: ");
                double length = input.nextDouble();

                double area = width * length;
                System.out.printf("The area of rectangle with width of %.2f and length of %.2f is %.2f", width, length, area);
            }
            static void calculateAreaOfCircle() {
                double pi = Math.PI;
                System.out.print("Enter length of the radius: ");
                double radius = input.nextDouble();

                double area = pi * radius * radius;
                System.out.printf("The area of circle with radius of %.2f is %.2f", radius, area);
            }
            public static void main(String[] args) {
                menu();
            }
        }

    \end{minted}
\end{enumerate}

\end{document}