\documentclass[12pt,titlepage]{article}
\usepackage[margin=1.25in]{geometry}
\usepackage{graphicx,amsmath,blindtext,minted}

%% Variables definition
\newcommand{\vSubject}{Data Structure and Algorithm Practicum}
\newcommand{\vSubtitle}{Brute Force and Divide Conquer}
\newcommand{\vName}{Muhammad Baihaqi Aulia Asy'ari}
\newcommand{\vNIM}{2241720145}
\newcommand{\vClass}{1I}
\newcommand{\vDepartment}{Information Technology}
\newcommand{\vStudyProgram}{D4 Informatics Engineering}

%% [START] Tikz related stuff
\usepackage{tikz}
\usetikzlibrary{svg.path,calc,shapes.geometric,shapes.misc}
\tikzstyle{terminator} = [rectangle, draw, text centered, rounded corners = 1em, minimum height=2em]
\tikzstyle{preparation} = [chamfered rectangle, chamfered rectangle sep=0.75em, draw, text centered, minimum height = 2em]
\tikzstyle{process} = [rectangle, draw, text centered, minimum height=2em]
\tikzstyle{decision} = [diamond, aspect=2, draw, text centered, minimum height=2em]
\tikzstyle{data}=[trapezium, draw, text centered, trapezium left angle=60, trapezium right angle=120, minimum height=2em]
\tikzstyle{connector} = [line width=0.25mm,->]
%% [END] Tikz related stuff

%% [START] Fancy header related stuff
\usepackage{fancyhdr}
\pagestyle{fancy}
\setlength{\headheight}{15pt} % compensate fancyhdr style
\fancyhead{}
\fancyfoot{}
\fancyfoot[L]{\thepage}
\fancyfoot[R]{\textit{\vSubject - \vSubtitle}}
\renewcommand{\footrulewidth}{0.4pt}% default is 0pt, overline for footer
%% [END] Fancy header related stuff

%% [START] Custom tabular command related stuff
\usepackage{tabularx}
\newcommand{\details}[2]{
    #1 & #2  \\
}
%% [END] Custom tabular command related stuff

%% [START] Figure related stuff
\newcommand{\image}[3][1]{
    \begin{figure}[h]
        \centering
        \includegraphics[#1]{#2}
        \caption{#3}
        \label{#3}
    \end{figure}
}
%% [END] Figure related stuff

%%
\usepackage{pgf-umlcd}

\renewcommand{\umldrawcolor}{black}
\renewcommand{\umlfillcolor}{white}
%%

%%
\usepackage{pdfpages}
%%

\begin{document}
\begin{titlepage}
    \centering
    \vfill
    {\bfseries\LARGE
        \vSubject\\
        \vskip0.25cm
        \vSubtitle
    }
    \vfill
    \includegraphics[width=6cm]{images/polinema-logo.png}
    \vfill
    {
        \textbf{Name}\\
        \vName\\
        \vskip0.5cm
        \textbf{NIM}\\
        \vNIM\\
        \vskip0.5cm
        \textbf{Class}\\
        \vClass\\
        \vskip0.5cm
        \textbf{Department}\\
        \vDepartment\\
        \vskip0.5cm
        \textbf{Study Program}\\
        \vStudyProgram
    }
\end{titlepage}

\newpage

\setcounter{section}{2}
\setcounter{subsection}{2}
\subsection{Calculating Factorial Values with Brute Force and Divide and Conquer Algorithms}
\subsubsection{Practicum}

\begin{enumerate}
    \item Create a new Project, with the name AlgoStruDat / Project Name Equated to last week. Make a package with the name \textbf{Week3}, make a new class with the name \textbf{Faktorial}.
    \item Complete the Faktorial class with the attributes and methods described in the class diagram above:
    \begin{enumerate}
        \item Add value attributes
        \begin{minted}[autogobble,breaklines]{java}
            public int num;
        \end{minted}
        \item Add method \texttt{faktorialBF()}
        \begin{minted}[autogobble,breaklines]{java}
            public int faktorialBF(int n) {
                int fakto = 1;
                for (int i = 1; i <= n; i++) {
                    fakto = fakto * i;
                }
                return fakto;
            }
        \end{minted}
        \item Add method \texttt{faktorialDC()}
        \begin{minted}[autogobble,breaklines]{java}
            public int faktorialDC(int n) {
                if (n==1) {
                    return 1;
                }
                else
                {
                    int fakto = n * faktorialDC(n-1);
                    return fakto;
                }
            }
        \end{minted}
    \end{enumerate}
    \item Run the \texttt{Faktorial class} y creating a new \texttt{MainFaktorial} class.
    \begin{enumerate}
        \item In the main function, provide input to input the number of numbers to find the factorial value
        \begin{minted}[autogobble,breaklines]{java}
            Scanner sc = new Scanner(System.in);
            System.out.println("================================ ================================");
            System.out.print("Input the number of elements you want to count : ");
            int elemen = sc.nextInt();
        \end{minted}
        \item Create an Array of Objects on the main function, then input some values that will be factorially calculated
        \begin{minted}[autogobble,breaklines]{java}
            Faktorial [] fk = new Faktorial[elemen];
            for (int i = 0; i < elemen; i++) {
                fk[i] = new Factorial();
                System.out.print("Input the data value to-"+(i+1)+" : ");
                fk[i].num = sc.nextInt();
            }
        \end{minted}
        \item Display the results of calling method \texttt{faktorialDC()} dan \texttt{faktorialBF()}
        \begin{minted}[autogobble,breaklines]{java}
            System.out.println("================================ ================================");
            System.out.println("Factorial Result with Brute Force");
            for (int i = 0; i < elemen; i++) {
                System.out.println("Factorial of value"+fk[i].num+" is : "+fk[i].faktorialBF(fk[i].num));
            }
            
            System.out.println("================================ ================================");
            for (int i = 0; i < elemen; i++) {
                System.out.println("Factorial of value"+fk[i].num+" is : "+fk[i].faktorialDC(fk[i].num));
            }
            System.out.println("================================ ================================");
        \end{minted}
        \item Make sure the program is running well!
    \end{enumerate}
\end{enumerate}

\begin{enumerate}
    \item \texttt{Faktorial.java}
    \begin{minted}[autogobble,breaklines]{java}
        package Faktorial;

        public class Faktorial {
            public int num;
            public int faktorialBF(int n) {
                int fakto = 1;
                for (int i = 1; i <= n; i++) {
                    fakto = fakto * i;
                }
                return fakto;
            }
            public int faktorialDC(int n) {
                if (n==1) {
                    return 1;
                }
                else
                {
                    int fakto = n * faktorialDC(n-1);
                    return fakto;
                }
            }
        }
    \end{minted}
    \item \texttt{MainFaktorial.java}
    \begin{minted}[autogobble,breaklines]{java}
        package Faktorial;

        public class MainFaktorial {
            public static void main(String[] args) {
                Scanner sc = new Scanner(System.in);
                    System.out.println("================================ ================================");
                    System.out.print("Input the number of elements you want to count : ");
                    int elemen = sc.nextInt();
                    Faktorial [] fk = new Faktorial[elemen];
                    for (int i = 0; i < elemen; i++) {
                        fk[i] = new Factorial();
                        System.out.print("Input the data value to-"+(i+1)+" : ");
                        fk[i].num = sc.nextInt();
                    }
                    System.out.println("================================ ================================");
                    System.out.println("Factorial Result with Brute Force");
                    for (int i = 0; i < elemen; i++) {
                        System.out.println("Factorial of value "+fk[i].num+" is : "+fk[i].faktorialBF(fk[i].num));
                    }
                    
                    System.out.println("================================ ================================");
                    for (int i = 0; i < elemen; i++) {
                        System.out.println("Factorial of value "+fk[i].num+" is : "+fk[i].faktorialDC(fk[i].num));
                    }
                    System.out.println("================================ ================================");
            }
        }
    \end{minted}
\end{enumerate}

\subsubsection{Verification of Practicum Results}

\begin{minted}[autogobble,breaklines]{text}
    ================================================================
    Input the number of elements you want to count : 3
    Input the data value to-1 : 5
    Input the data value to-2 : 8
    Input the data value to-3 : 3
    ================================================================
    Factorial Results with Brute Force
    Factorial of value 5 : 120
    Factorial of value 8 : 40320
    Factorial of value 3 : 6
    ================================================================
    Factorial Results with Divide and Conquer
    Factorial of value 5 : 120
    Factorial of value 8 : 40320
    Factorial of value 3 : 6
    ================================================================
\end{minted}

\texttt{Result: }

\begin{minted}[autogobble,breaklines,linenos]{text}
    PS D:\Kuliah>  d:; cd 'd:\Kuliah'; & 'C:\Program Files\Java\jdk-18.0.2.1\bin\java.exe' '-XX:+ShowCodeDetailsInExceptionMessages' '-cp' 'C:\Users\ASUS\AppData\Roaming\Code\User\workspaceStorage\ ce3fcb236261368a6cbd019dc8ddda8b\redhat.java\jdt_ws\ Kuliah_28156aa7\bin' 'Faktorial.MainFaktorial' 
    ================================================================
    Input the number of elements you want to count : 3
    Input the data value to-1 : 5
    Input the data value to-2 : 8
    Input the data value to-3 : 3
    ================================================================
    Factorial Result with Brute Force
    Factorial of value 5 is : 120
    Factorial of value 8 is : 40320
    Factorial of value 3 is : 6
    ================================================================
    Factorial of value 5 is : 120
    Factorial of value 8 is : 40320
    Factorial of value 3 is : 6
    ================================================================
\end{minted}

\subsubsection{Questions}

\begin{enumerate}
    \item Explain the Divide Conquer Algorithm for calculating factorial values!
    \mbox{}\\ with the 
    \item In the implementation of Factorial Divide and Conquer Algorithm is it complete that consists of 3 stages of divide, conquer, combine? Explain each part of the program code!
    \item Is it possible to repeat the factorial BF () method instead of using for? Prove it!
    \item Add a check to the execution time of the two types of methods!
    \item Prove by inputting elements that are above 20 digits, is there a difference in execution time?
\end{enumerate}

\end{document}