\documentclass[12pt,titlepage]{article}
\usepackage[margin=1.25in]{geometry}
\usepackage{graphicx,amsmath,blindtext,minted}

%% Variables definition
\newcommand{\vSubject}{Data Structure and Algorithm Practicum}
\newcommand{\vSubtitle}{Chapter}
\newcommand{\vName}{Muhammad Baihaqi Aulia Asy'ari}
\newcommand{\vNIM}{2241720145}
\newcommand{\vClass}{1I}
\newcommand{\vDepartment}{Information Technology}
\newcommand{\vStudyProgram}{D4 Informatics Engineering}

%% [START] Tikz related stuff
\usepackage{tikz}
\usetikzlibrary{svg.path,calc,shapes.geometric,shapes.misc}
\tikzstyle{terminator} = [rectangle, draw, text centered, rounded corners = 1em, minimum height=2em]
\tikzstyle{preparation} = [chamfered rectangle, chamfered rectangle sep=0.75em, draw, text centered, minimum height = 2em]
\tikzstyle{process} = [rectangle, draw, text centered, minimum height=2em]
\tikzstyle{decision} = [diamond, aspect=2, draw, text centered, minimum height=2em]
\tikzstyle{data}=[trapezium, draw, text centered, trapezium left angle=60, trapezium right angle=120, minimum height=2em]
\tikzstyle{connector} = [line width=0.25mm,->]
%% [END] Tikz related stuff

%% [START] Fancy header related stuff
\usepackage{fancyhdr}
\pagestyle{fancy}
\setlength{\headheight}{15pt} % compensate fancyhdr style
\fancyhead{}
\fancyfoot{}
\fancyfoot[L]{\thepage}
\fancyfoot[R]{\textit{\vSubject - \vSubtitle}}
\renewcommand{\footrulewidth}{0.4pt}% default is 0pt, overline for footer
%% [END] Fancy header related stuff

%% [START] Custom tabular command related stuff
\usepackage{tabularx}
\newcommand{\details}[2]{
    #1 & #2  \\
}
%% [END] Custom tabular command related stuff

%% [START] Figure related stuff
\newcommand{\image}[3][1]{
    \begin{figure}[h]
        \centering
        \includegraphics[#1]{#2}
        \caption{#3}
        \label{#3}
    \end{figure}
}
%% [END] Figure related stuff

%%
\usepackage{pgf-umlcd}

\renewcommand{\umldrawcolor}{black}
\renewcommand{\umlfillcolor}{white}
%%

%% [BEGIN] Custom enumerator
\usepackage{enumitem}
%% [END] Custom enumerator

%% [BEGIN] Paragraph indent
\usepackage{indentfirst}
%% [END] Paragraph indent

\begin{document}
\begin{titlepage}
    \centering
    \vfill
    {\bfseries\LARGE
        \vSubject\\
        \vskip0.25cm
        \vSubtitle
    }
    \vfill
    \includegraphics[width=6cm]{images/polinema-logo.png}
    \vfill
    {
        \textbf{Name}\\
        \vName\\
        \vskip0.5cm
        \textbf{NIM}\\
        \vNIM\\
        \vskip0.5cm
        \textbf{Class}\\
        \vClass\\
        \vskip0.5cm
        \textbf{Department}\\
        \vDepartment\\
        \vskip0.5cm
        \textbf{Study Program}\\
        \vStudyProgram
    }
\end{titlepage}

\newpage

\setcounter{section}{16}
\subsection{Tujuan Praktikum}
Setelah melakukan praktikum ini, mahasiswa mampu:
\begin{enumerate}
    \item memahami bentuk-bentuk collection dan hierarkinya;
    \item menerapkan collection sesuai dengan fungsi dan jenisnya;
    \item menyelesaikan kasus menggunakan collection yang sesuai.
\end{enumerate}

\subsection{Kegiatan Praktikum 1}
\subsubsection{Percobaan 1}
Pada percobaan 1 ini akan dicontohkan penggunaan collection untuk menambahkan sebuah elemen, mengakses elemen, dan menghapus sebuah elemen.
\begin{enumerate}
    \item Buatlah sebuah class ContohList yang main methode berisi kode program seperti di bawah ini
    \begin{minted}[autogobble,fontsize=\small,breaklines]{java}
        List l = new ArrayList<>();
        l.add(1);
        l.add(2);
        l.add(3);
        l.add("Cireng");
        System.out.printf("Elemen 0: %d total elemen: %d elemen terakhir: %s\n", l.get(0), l.size(), l.get(l.size() - 1));
        
        l.add(4);
        l.remove(0);
        System.out.printf("Elemen 0: %d total elemen: %d elemen terakhir: %s\n", l.get(0), l.size(), l.get(l.size() - 1));
    \end{minted}
    \item Tambahkan kode program untuk menggunakan collection dengan aturan penulisan kode program seperti berikut
    \begin{minted}[autogobble,fontsize=\small,breaklines]{java}
        List<String> names = new LinkedList<>();
        names.add("Noureen");
        names.add("Akhleema");
        names.add("Shannum");
        names.add("Uwais");
        names.add("Al-Qarni");
        
        System.out.printf("Elemen 0: %s total elemen: %d elemen terakhir: %s\n", names.get(0), names.size(), l.get(l.size() - 1));
        names.set(0, "My Kid");
        System.out.printf("Elemen 0: %s total elemen: %d elemen terakhir: %s\n", names.get(0), names.size(), l.get(l.size() - 1));
        System.out.println("Names: " + names.toString());
    \end{minted}
\end{enumerate}

\subsubsection{Verifikasi Hasil Percobaan}
Verifikasi hasil kompilasi kode program Anda dengan gambar berikut ini.
\begin{minted}[autogobble,breaklines,linenos]{text}
    PS D:\Kuliah\Smt 2\Algoritma dan Struktur Data\Praktikum\Week 16\Collection>  & 'C:\Program Files\Java\jdk-18.0.2.1\bin\java.exe' '-XX:+ShowCodeDetailsInExceptionMessages' '-cp' 'D:\Kuliah\Smt 2\Algoritma dan Struktur Data\Praktikum\Week 16\Collection\bin' 'ContohList'
    Elemen 0: 1 total elemen: 4 elemen terakhir: Cireng 
    Elemen 0: 2 total elemen: 4 elemen terakhir: 4      
    Elemen 0: Noureen total elemen: 5 elemen terakhir: 4
    Elemen 0: My Kid total elemen: 5 elemen terakhir: 4 
    Names: [My Kid, Akhleema, Shannum, Uwais, Al-Qarni]
\end{minted}

\subsubsection{Pertanyaan Percobaan}
\begin{enumerate}
    \item Perhatikan baris kode 25-36, mengapa semua jenis data bisa ditampung ke dalam sebuah Arraylist?
    \item Modifikasi baris kode 25-36 seingga data yang ditampung hanya satu jenis atau spesifik tipe tertentu!
    \item Ubah kode pada baris kode 38 menjadi seperti ini
    \item Tambahkan juga baris berikut ini, untuk memberikan perbedaan dari tampilan yang sebelumnya
    \item Dari penambahan kode tersebut, silakan dijalankan dan apakah yang dapat Anda jelaskan!
\end{enumerate}

\subsection{Kegiatan Praktikum 2}
\subsubsection{Tahapan Percobaan}
Pada praktikum 2 ini akan dibuat beberapa method untuk menampilkan beberapa cara yang dapat dilakukan untuk mengambil/menampilkan elemen pada sebuah collection. Silakan ikutilah Langkah-langkah di bawah ini
\begin{enumerate}
    \item Buatlah class dengan nama LoopCollection serta tambahkan method main yang isinya adalah sebagai berikut.
    \item Tambahkan potongan kode berikut ini dari yang sebelumnya agar proses menampilkan elemen pada sebuah stack bervariasi.
\end{enumerate}

\subsubsection{Verifikasi Hasil Percobaan}
Verifikasi hasil kompilasi kode program Anda dengan gambar berikut ini.

\subsubsection{Pertanyaan Percobaan}
\begin{enumerate}
    \item Apakah perbedaan fungsi push() dan add() pada objek fruits?
    \item Silakan hilangkan baris 43 dan 44, apakah yang akan terjadi? Mengapa bisa demikian?
    \item Jelaskan fungsi dari baris 46-49?
    \item Silakan ganti baris kode 25, Stack<String> menjadi List<String> dan apakah yang terjadi? Mengapa bisa demikian?
    \item Ganti elemen terakhir dari dari objek fruits menjadi “Strawberry”!
    \item Tambahkan 3 buah seperti “Mango”,”guava”, dan “avocado” kemudian dilakukan sorting!
\end{enumerate}

\subsection{Kegiatan Praktikum 3}
\subsubsection{Tahapan Percobaan}
Pada praktikum 3 ini dilakukan uji coba untuk mengimplementasikan sebuah collection untuk menampung objek yang dibuat sesuai kebutuhan. Objek tersebut adalah sebuah objek mahasiswa dengan fungsi-fungsi umum seperti menambahkan, menghapus, mengubah, dan mencari.
\begin{enumerate}
    \item Buatlah sebuah class Mahasiswa dengan attribute, kontruktor, dan fungsi sebagai berikut.
    \item Selanjutnya, buatlah sebuah class ListMahasiswa yang memiliki attribute seperti di bawah ini
    \item Method tambah(), hapus(), update(), dan tampil() secara berurut dibuat agar bisa melakukan operasi-operasi seperti yang telah disebutkan.
    \item Untuk proses hapus, update membutuhkan fungsi pencarian terlebih dahulu yang potongan kode programnya adalah sebagai berikut
    \item Pada class yang sama, tambahkan main method seperti potongan program berikut dan amati hasilnya!
\end{enumerate}

\subsubsection{Verifikasi Hasil Percobaan}
Verifikasi hasil kompilasi kode program Anda dengan gambar berikut ini.

\subsubsection{Pertanyaan Percobaan}
\begin{enumerate}
    \item Pada fungsi tambah() yang menggunakan unlimited argument itu menggunakan konsep apa? Dan kelebihannya apa?
    \item Pada fungsi linearSearch() di atas, silakan diganti dengan fungsi binarySearch() dari collection!
    \item Tambahkan fungsi sorting baik secara ascending ataupun descending pada class tersebut!
\end{enumerate}

\subsection{Tugas Praktikum}
\begin{enumerate}
    \item Buatlah implementasi program daftar nilai mahasiswa semester, minimal memiliki 3 class yaitu Mahasiswa, Nilai, dan Mata Kuliah. Data Mahasiswa dan Mata Kuliah perlu melalui penginputan data terlebih dahulu.
    \mbox{}\\ \textbf{Ilustrasi Program} \mbox{}\\ 
    \textit{Menu Awal dan Penambahan Data}
    \textit{Tampil Nilai}
    \textit{Pencarian Data Mahasiswa}
    \textit{Pengurutan Data Nilai}
    \item Tambahkan prosedur hapus data mahasiswa melalui implementasi Queue pada collections Tugas nomor 1!
\end{enumerate}

\end{document}