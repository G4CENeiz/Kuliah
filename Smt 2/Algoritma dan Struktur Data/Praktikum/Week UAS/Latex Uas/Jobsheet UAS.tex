\documentclass[12pt,titlepage]{article}
\usepackage[margin=1.25in]{geometry}
\usepackage{graphicx,amsmath,blindtext,minted}

%% Variables definition
\newcommand{\vSubject}{Data Structure and Algorithm Practicum}
\newcommand{\vSubtitle}{UJIAN AKHIR SEMESTER GENAP TAHUN AKADEMIK 2022/2023}
\newcommand{\vName}{Muhammad Baihaqi Aulia Asy'ari}
\newcommand{\vNIM}{2241720145}
\newcommand{\vClass}{1I}
\newcommand{\vDepartment}{Information Technology}
\newcommand{\vStudyProgram}{D4 Informatics Engineering}

%% [START] Tikz related stuff
\usepackage{tikz}
\usetikzlibrary{svg.path,calc,shapes.geometric,shapes.misc}
\tikzstyle{terminator} = [rectangle, draw, text centered, rounded corners = 1em, minimum height=2em]
\tikzstyle{preparation} = [chamfered rectangle, chamfered rectangle sep=0.75em, draw, text centered, minimum height = 2em]
\tikzstyle{process} = [rectangle, draw, text centered, minimum height=2em]
\tikzstyle{decision} = [diamond, aspect=2, draw, text centered, minimum height=2em]
\tikzstyle{data}=[trapezium, draw, text centered, trapezium left angle=60, trapezium right angle=120, minimum height=2em]
\tikzstyle{connector} = [line width=0.25mm,->]
%% [END] Tikz related stuff

%% [START] Fancy header related stuff
\usepackage{fancyhdr}
\pagestyle{fancy}
\setlength{\headheight}{15pt} % compensate fancyhdr style
\fancyhead{}
\fancyfoot{}
\fancyfoot[L]{\thepage}
\fancyfoot[R]{\textit{\vSubject - \vSubtitle}}
\renewcommand{\footrulewidth}{0.4pt}% default is 0pt, overline for footer
%% [END] Fancy header related stuff

%% [START] Custom tabular command related stuff
\usepackage{tabularx}
\newcommand{\details}[2]{
    #1 & #2  \\
}
%% [END] Custom tabular command related stuff

%% [START] Figure related stuff
\newcommand{\image}[3][1]{
    \begin{figure}[h]
        \centering
        \includegraphics[#1]{#2}
        \caption{#3}
        \label{#3}
    \end{figure}
}
%% [END] Figure related stuff

%%
\usepackage{pgf-umlcd}

\renewcommand{\umldrawcolor}{black}
\renewcommand{\umlfillcolor}{white}
%%

%% [BEGIN] Custom enumerator
\usepackage{enumitem}
%% [END] Custom enumerator

%% [BEGIN] Paragraph indent
\usepackage{indentfirst}
%% [END] Paragraph indent

\begin{document}
\begin{titlepage}
    \centering
    \vfill
    {\bfseries\LARGE
        \vSubject\\
        \vskip0.25cm
        \vSubtitle
    }
    \vfill
    \includegraphics[width=6cm]{images/polinema-logo.png}
    \vfill
    {
        \textbf{Name}\\
        \vName\\
        \vskip0.5cm
        \textbf{NIM}\\
        \vNIM\\
        \vskip0.5cm
        \textbf{Class}\\
        \vClass\\
        \vskip0.5cm
        \textbf{Department}\\
        \vDepartment\\
        \vskip0.5cm
        \textbf{Study Program}\\
        \vStudyProgram
    }
\end{titlepage}

\newpage

\section*{Soal UAS}
Telah disediakan sebuah project tentang Double Linked List yang terdiri dari 3 class, yaitu class \textbf{Node2P}, class \textbf{DLL} dan class \textbf{DLLMain}. Project tersebut dibuat menggunakan Netbeans, jika Anda menggunakan Netbeans bisa langsung dibuka project tersebut, jika Anda menggunakan tools yang lain, bisa diambil saja class-class nya. Fungsi-fungsi utama terkait Double Linked List telah disediakan di dalam class DLL, seperti \textbf{isEmpty(), size(), addLast(), deleteFirst(), deleteLast(), deleteByData(), getDataByIndex() dan print()}. Method deleteByData() digunakan untuk menghapus elemen dari list berdasarkan nilai datanya. Sedangkan method getDataByIndex() digunakan untuk mengambil nilai data berdasarkan posisinya.

\section*{Pertanyaan}
Lengkapi kode program di class \textbf{DLL}, untuk method-method di bawah ini:
\begin{enumerate}
    \item \textbf{toArray()}: method ini akan menghasilkan sebuah variable array, dimana datanya berasal dari data di dalam objek linked list yang sudah ada.
    \begin{minted}[autogobble,breaklines]{java}
        int[] toArray(){
            int dataCount = 0;
            Node2P current = head;
            if (isEmpty()) {
                dataCount = 0;
                System.out.println("list is empty");
            } else {
                dataCount = 1;
                do {
                    dataCount++;
                    current = current.next;
                } while (current.next != null);
            }
            int[] data = new int[dataCount];
            for (int i = 0; i < dataCount; i++) {
                data[i] = get(i);
            }
            return data;
        }
    \end{minted}
    \item \textbf{sublist(int start, int end)}: method ini digunakan untuk mengembalikan list baru yang mengambil sebagian dari data yang sudah ada di list dari posisi \textbf{start} sampai posisi \textbf{end}
    \begin{minted}[autogobble,breaklines]{java}
        DLL sublist(int start, int end){
            Node2P current = head;
            DLL data = new DLL();
            for (int i = 0; i <= end; i++) {
                if (i >= start) {
                    data.addLast(get(i));
                }
                current = current.next;
            }
            return data;
        }
    \end{minted}
    \item \textbf{addAll(DLL list)}: method ini digunakan untuk menambahkan data yang ada di \textbf{list} ke dalam list yang sudah ada
    \begin{minted}[autogobble,breaklines]{java}
        void addAll(DLL list){
            Node2P currentList = list.head;
            if (isEmpty()) {
                System.out.println("List is empty");
            } else {
                do {
                    addLast(currentList.data);
                    currentList = currentList.next;
                } while (currentList != null);
            }
        }
    \end{minted}
    \item \textbf{containsAll(DLL list)}: method ini akan mengecek apakah semua data yang ada di dalam list, ada di dalam list yang sudah ada
    \begin{minted}[autogobble,breaklines]{java}
        boolean containsAll(DLL list){
            int[] thisList = toArray();
            int[] inputList = list.toArray();
            boolean bool3 = false;
            boolean[] bool2 = new boolean[inputList.length];
            for (int i = 0; i < inputList.length; i++) {
                boolean[] bool = new boolean[thisList.length];
                for (int j = 0; j < thisList.length; j++) {
                    bool[j] = thisList[j] == inputList[i];
                }
                for (int j = 0; j < bool2.length; j++) { 
                    for (boolean b : bool) {
                        if (b) {
                            bool2[j] = b;
                        }
                    }
                }
            }
            for (boolean b : bool2) {
                if (b) {
                    bool3 = b;
                }
            }
            return bool3;
        }
    \end{minted}
    \item \textbf{removeAll(DLL list)}: method ini akan menghapus data dari dalam list yang sudah ada berdasarkan nilai yang ada di dalam \textbf{list}
    \begin{minted}[autogobble,breaklines]{java}
        void removeAll(DLL list){
            Node2P currentList = list.head;
            if (isEmpty()) {
                System.out.println("list is empty");
            } else {
                do {
                    deleteByData(currentList.data);
                    currentList = currentList.next;
                } while (currentList != null);
            }
        }
    \end{minted}
\end{enumerate}
Jika class DLLMain dijalankan, maka akan didapatkan hasil seperti di bawah ini:
\begin{minted}[autogobble,breaklines,linenos]{text}
    Tampilan data awal DLL:
    10-100-34-20-200-75
    Tampilan data array hasil dari fungsi toArray():
    10, 100, 34, 20, 200, 75,
    Tampilan data dari list hasil dari fungsi sublist(1,3):
    100-34-20
    Tampilan data dari list hasil dari fungsi addAll():
    10-100-34-20-200-75-212-212-212
    Tampilan data dari fungsi containsAll():
    true
    Tampilan data dari fungsi removeAll():
    10-20-200-75
\end{minted}

\end{document}