\documentclass[12pt,titlepage]{article}
\usepackage[margin=1.25in]{geometry}
\usepackage{graphicx,amsmath,blindtext,minted}

%% Variables definition
\newcommand{\vSubject}{Data Structure and Algorithm Practicum}
\newcommand{\vSubtitle}{Searching}
\newcommand{\vName}{Muhammad Baihaqi Aulia Asy'ari}
\newcommand{\vNIM}{2241720145}
\newcommand{\vClass}{1I}
\newcommand{\vDepartment}{Information Technology}
\newcommand{\vStudyProgram}{D4 Informatics Engineering}

%% [START] Tikz related stuff
\usepackage{tikz}
\usetikzlibrary{svg.path,calc,shapes.geometric,shapes.misc}
\tikzstyle{terminator} = [rectangle, draw, text centered, rounded corners = 1em, minimum height=2em]
\tikzstyle{preparation} = [chamfered rectangle, chamfered rectangle sep=0.75em, draw, text centered, minimum height = 2em]
\tikzstyle{process} = [rectangle, draw, text centered, minimum height=2em]
\tikzstyle{decision} = [diamond, aspect=2, draw, text centered, minimum height=2em]
\tikzstyle{data}=[trapezium, draw, text centered, trapezium left angle=60, trapezium right angle=120, minimum height=2em]
\tikzstyle{connector} = [line width=0.25mm,->]
%% [END] Tikz related stuff

%% [START] Fancy header related stuff
\usepackage{fancyhdr}
\pagestyle{fancy}
\setlength{\headheight}{15pt} % compensate fancyhdr style
\fancyhead{}
\fancyfoot{}
\fancyfoot[L]{\thepage}
\fancyfoot[R]{\textit{\vSubject - \vSubtitle}}
\renewcommand{\footrulewidth}{0.4pt}% default is 0pt, overline for footer
%% [END] Fancy header related stuff

%% [START] Custom tabular command related stuff
\usepackage{tabularx}
\newcommand{\details}[2]{
    #1 & #2  \\
}
%% [END] Custom tabular command related stuff

%% [START] Figure related stuff
\newcommand{\image}[3][1]{
    \begin{figure}[h]
        \centering
        \includegraphics[#1]{#2}
        \caption{#3}
        \label{#3}
    \end{figure}
}
%% [END] Figure related stuff

%%
\usepackage{pgf-umlcd}

\renewcommand{\umldrawcolor}{black}
\renewcommand{\umlfillcolor}{white}
%%

%%
\usepackage{pdfpages}
%%

\begin{document}
\begin{titlepage}
    \centering
    \vfill
    {\bfseries\LARGE
        \vSubject\\
        \vskip0.25cm
        \vSubtitle
    }
    \vfill
    \includegraphics[width=6cm]{images/polinema-logo.png}
    \vfill
    {
        \textbf{Name}\\
        \vName\\
        \vskip0.5cm
        \textbf{NIM}\\
        \vNIM\\
        \vskip0.5cm
        \textbf{Class}\\
        \vClass\\
        \vskip0.5cm
        \textbf{Department}\\
        \vDepartment\\
        \vskip0.5cm
        \textbf{Study Program}\\
        \vStudyProgram
    }
\end{titlepage}

\newpage

\setcounter{section}{1}
\setcounter{subsection}{1}
\subsection{Sequential Search Method}
\subsubsection{Steps}

\begin{enumerate}
    \item Create a new project in NetBeans called \texttt{TestSearching}
    \item Then, create a new package \texttt{week7}.
    \item Create new \texttt{Students} class, then declare following attributes:
    \begin{minted}[autogobble, breaklines]{java}
        public class Students {
            int nim, age;
            String name;
            double gpa;
        }
    \end{minted}
    \item Create a constructor in \texttt{Students} class with parameters (int ni, String nm, int age, double gpa). Convert it to program code as follows:
    \begin{minted}[autogobble,breaklines]{java}
        public Students(int nim, int age, String name, double gpa) {
            this.nim = nim;
            this.age = age;
            this.name = name;
            this.gpa = gpa;
        }
    \end{minted}
    \item Create display() method with void as its return type
    \begin{minted}[autogobble,breaklines]{java}
        public void display() {
            System.out.println("NIM =" + nim);
            System.out.println("Name =" + name);
            System.out.println("Age =" + age);
            System.out.println("GPA =" + gpa);
        }
    \end{minted}
    \item Create a new \texttt{SearchStudent} class as follows.
    \begin{minted}[autogobble,breaklines]{java}
        public class SearchStudent {
            Students[] listStd = new Students[5];
            int idx;
        }
    \end{minted}
    \item Create method \texttt{add()} at that class! This will be used for adding objects from \texttt{Students} class to listStd attribute
    \begin{minted}[autogobble,breaklines]{java}
        public void add(Students std) {
            if (idx < listStd.length) {
                listStd[idx] = std;
                idx++;
            } else {
                System.out.println("Data is already full");
            }
        }
    \end{minted}
    \item Create method \texttt{display()} in class \texttt{SearchStudent}! This display() method will be used to print all students data available in this class. Pay attention on how we use \texttt{for loops} differently. Even so, the concepts is still the same
    \begin{minted}[autogobble,breaklines]{java}
        public void display() {
            for (Students students : listStd) {
                students.display();
                System.out.println("--------------------------------");
            }
        }
    \end{minted}
    \item Create method \texttt{FindSeqSearch} with integer as its return type. Then fill in the function with sequential search algorithm.
    \begin{minted}[autogobble,breaklines]{java}
        public int findSeqSearch(int search) {
            int potition = 1;
            for (int i = 0; i < listStd.length; i++) {
                if (listStd[i].nim == search) {
                    potition = i;
                    break;
                }
            }
            return potition;
        }
    \end{minted}
    \item Create method \texttt{displayPosition} with void as its return type. And write these following code as follows
    \begin{minted}[autogobble,breaklines]{java}
        public void showPotition(int x, int pos) {
            if (pos != 1) {
                System.out.println("Data : " + x + " is found in index-" + pos);
            } else {
                System.out.println("Data : " + x + "is not found");
            }
        }
    \end{minted}
    \item Create method \texttt{displayData} with void as its return type. And write these following code as follows
    \begin{minted}[autogobble,breaklines]{java}
        public void showData(int x, int pos) {
            if (pos != 1) {
                System.out.println("NIM \t : " + x);
                System.out.println("Name \t : " + listStd[pos].name);
                System.out.println("Age \t : " + listStd[pos].age);
                System.out.println("IPK \t : " + listStd[pos].gpa);
            } else {
                System.out.println("Data " + x + " is not found");
            }
        }
    \end{minted}
    \item Create a main class named \texttt{StudentsMain} and add main method as follows
    \begin{minted}[autogobble,breaklines]{java}
        public class StudentsMain {
            public static void main(String[] args) {
            
            }
        }
    \end{minted}
    \item In main method, instantiate an object in \texttt{SearchStudent} that consist of \texttt{5 Students}, then add all students object by calling \texttt{add} function in object \texttt{SearchStudent}
    \begin{minted}[autogobble,breaklines]{java}
        Scanner s = new Scanner(System.in);
        Scanner sl = new Scanner(System.in);

        SearchStudent data = new SearchStudent();
        int amountStudent = 5;

        System.out.println("--------------------------------");
        System.out.println("Input student data accordingly from samllest NIM");
        for (int i = 0; i < amountStudent; i++) {
            System.out.println("----------------");
            System.out.print("NIM\t: ");
            int nim = s.nextInt();
            System.out.print("Name\t: ");
            String name = sl.nextLine();
            System.out.print("Age\t: ");
            int age = s.nextInt();
            System.out.print("GPA\t: ");
            double gpa = s.nextDouble();

            Students std = new Students(nim, age, name, gpa);
            data.add(std);
        }
    \end{minted}
    \item Add method \texttt{display} to print all inserted data
    \begin{minted}[autogobble,breaklines]{java}
        System.out.println("--------------------------------");
        System.out.println("Entire Student Data");
        data.display();
    \end{minted}
    \item To search students by their NIM, create a search variable to hold input from user. Then call method FindSeqSearch with its parameter is the search variable we’ve declared before
    \begin{minted}[autogobble,breaklines]{java}
        System.out.println("________________________");
        System.out.println("________________________");
        System.out.print("Search student by NIM: ");
        int search = s.nextInt();
        System.out.println("Using Sequential Search");
        int potition = data.findSeqSearch(search);
    \end{minted}
    \item Call method displayPosition from class SearchStudent.
    \begin{minted}[autogobble,breaklines]{java}
        data.showPotition(search, potition);
    \end{minted} 
    \item Call method displayData from class SearchStudent
    \begin{minted}[autogobble,breaklines]{java}
        data.showData(search, potition);
    \end{minted}
    \item Run the program and see the result
    \begin{minted}[autogobble,breaklines]{java}
        package week7;

        public class Students {
            int nim, age;
            String name;
            double gpa;

            public Students(int nim, int age, String name, double gpa) {
                this.nim = nim;
                this.age = age;
                this.name = name;
                this.gpa = gpa;
            }

            public void display() {
                System.out.println("NIM =" + nim);
                System.out.println("Name =" + name);
                System.out.println("Age =" + age);
                System.out.println("GPA =" + gpa);
            }
        }
    \end{minted}
    \begin{minted}[autogobble,breaklines]{java}
        package week7;

        public class SearchStudent {
            Students[] listStd = new Students[5];
            int idx;
            public void add(Students std) {
                if (idx < listStd.length) {
                    listStd[idx] = std;
                    idx++;
                } else {
                    System.out.println("Data is already full");
                }
            }

            public void display() {
                for (Students students : listStd) {
                    students.display();
                    System.out.println("--------------------------------");
                }
            }

            public int findSeqSearch(int search) {
                int potition = 1;
                for (int i = 0; i < listStd.length; i++) {
                    if (listStd[i].nim == search) {
                        potition = i;
                        break;
                    }
                }
                return potition;
            }

            public void showPotition(int x, int pos) {
                if (pos != 1) {
                    System.out.println("Data : " + x + " is found in index-" + pos);
                } else {
                    System.out.println("Data : " + x + "is not found");
                }
            }

            public void showData(int x, int pos) {
                if (pos != 1) {
                    System.out.println("NIM \t : " + x);
                    System.out.println("Name \t : " + listStd[pos].name);
                    System.out.println("Age \t : " + listStd[pos].age);
                    System.out.println("IPK \t : " + listStd[pos].gpa);
                } else {
                    System.out.println("Data " + x + " is not found");
                }
            }
        }
    \end{minted}
    \begin{minted}[autogobble,breaklines]{java}
        package week7;

        import java.util.Scanner;

        public class StudentsMain {
            public static void main(String[] args) {
                Scanner s = new Scanner(System.in);
                Scanner sl = new Scanner(System.in);

                SearchStudent data = new SearchStudent();
                int amountStudent = 5;

                System.out.println("--------------------------------");
                System.out.println("Input student data accordingly from samllest NIM");
                for (int i = 0; i < amountStudent; i++) {
                    System.out.println("----------------");
                    System.out.print("NIM\t: ");
                    int nim = s.nextInt();
                    System.out.print("Name\t: ");
                    String name = sl.nextLine();
                    System.out.print("Age\t: ");
                    int age = s.nextInt();
                    System.out.print("GPA\t: ");
                    double gpa = s.nextDouble();

                    Students std = new Students(nim, age, name, gpa);
                    data.add(std);
                }

                System.out.println("--------------------------------");
                System.out.println("Entire Student Data");
                data.display();

                System.out.println("________________________");
                System.out.println("________________________");
                System.out.print("Search student by NIM: ");
                int search = s.nextInt();
                System.out.println("Using Sequential Search");
                int potition = data.findSeqSearch(search);

                data.showPotition(search, potition);

                data.showData(search, potition);

                s.close();
                sl.close();
            }
        }
    \end{minted}
\end{enumerate}

\subsubsection{Result}

\end{document}