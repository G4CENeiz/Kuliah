\documentclass[12pt,titlepage]{article}
\usepackage[margin=1.25in]{geometry}
\usepackage{graphicx,amsmath,blindtext,minted}

%% Variables definition
\newcommand{\vSubject}{Data Structure and Algorithm Practicum}
\newcommand{\vSubtitle}{Searching}
\newcommand{\vName}{Muhammad Baihaqi Aulia Asy'ari}
\newcommand{\vNIM}{2241720145}
\newcommand{\vClass}{1I}
\newcommand{\vDepartment}{Information Technology}
\newcommand{\vStudyProgram}{D4 Informatics Engineering}

%% [START] Tikz related stuff
\usepackage{tikz}
\usetikzlibrary{svg.path,calc,shapes.geometric,shapes.misc}
\tikzstyle{terminator} = [rectangle, draw, text centered, rounded corners = 1em, minimum height=2em]
\tikzstyle{preparation} = [chamfered rectangle, chamfered rectangle sep=0.75em, draw, text centered, minimum height = 2em]
\tikzstyle{process} = [rectangle, draw, text centered, minimum height=2em]
\tikzstyle{decision} = [diamond, aspect=2, draw, text centered, minimum height=2em]
\tikzstyle{data}=[trapezium, draw, text centered, trapezium left angle=60, trapezium right angle=120, minimum height=2em]
\tikzstyle{connector} = [line width=0.25mm,->]
%% [END] Tikz related stuff

%% [START] Fancy header related stuff
\usepackage{fancyhdr}
\pagestyle{fancy}
\setlength{\headheight}{15pt} % compensate fancyhdr style
\fancyhead{}
\fancyfoot{}
\fancyfoot[L]{\thepage}
\fancyfoot[R]{\textit{\vSubject - \vSubtitle}}
\renewcommand{\footrulewidth}{0.4pt}% default is 0pt, overline for footer
%% [END] Fancy header related stuff

%% [START] Custom tabular command related stuff
\usepackage{tabularx}
\newcommand{\details}[2]{
    #1 & #2  \\
}
%% [END] Custom tabular command related stuff

%% [START] Figure related stuff
\newcommand{\image}[3][1]{
    \begin{figure}[h]
        \centering
        \includegraphics[#1]{#2}
        \caption{#3}
        \label{#3}
    \end{figure}
}
%% [END] Figure related stuff

%%
\usepackage{pgf-umlcd}

\renewcommand{\umldrawcolor}{black}
\renewcommand{\umlfillcolor}{white}
%%

%%
\usepackage{pdfpages}
%%

\begin{document}
\begin{titlepage}
    \centering
    \vfill
    {\bfseries\LARGE
        \vSubject\\
        \vskip0.25cm
        \vSubtitle
    }
    \vfill
    \includegraphics[width=6cm]{images/polinema-logo.png}
    \vfill
    {
        \textbf{Name}\\
        \vName\\
        \vskip0.5cm
        \textbf{NIM}\\
        \vNIM\\
        \vskip0.5cm
        \textbf{Class}\\
        \vClass\\
        \vskip0.5cm
        \textbf{Department}\\
        \vDepartment\\
        \vskip0.5cm
        \textbf{Study Program}\\
        \vStudyProgram
    }
\end{titlepage}

\newpage

\setcounter{section}{1}
\setcounter{subsection}{1}
\subsection{Sequential Search Method}
\subsubsection{Steps}

\begin{enumerate}
    \item Create a new project in NetBeans called \texttt{TestSearching}
    \item Then, create a new package \texttt{week7}.
    \item Create new \texttt{Students} class, then declare following attributes:
    \begin{minted}[autogobble, breaklines]{java}
        public class Students {
            int nim, age;
            String name;
            double gpa;
        }
    \end{minted}
    \item Create a constructor in \texttt{Students} class with parameters (int ni, String nm, int age, double gpa). Convert it to program code as follows:
    \begin{minted}[autogobble,breaklines]{java}
        public Students(int nim, int age, String name, double gpa) {
            this.nim = nim;
            this.age = age;
            this.name = name;
            this.gpa = gpa;
        }
    \end{minted}
    \item Create display() method with void as its return type
    \begin{minted}[autogobble,breaklines]{java}
        public void display() {
            System.out.println("NIM =" + nim);
            System.out.println("Name =" + name);
            System.out.println("Age =" + age);
            System.out.println("GPA =" + gpa);
        }
    \end{minted}
    \item Create a new \texttt{SearchStudent} class as follows.
    \begin{minted}[autogobble,breaklines]{java}
        public class SearchStudent {
            Students[] listStd = new Students[5];
            int idx;
        }
    \end{minted}
    \item Create method \texttt{add()} at that class! This will be used for adding objects from \texttt{Students} class to listStd attribute
    \begin{minted}[autogobble,breaklines]{java}
        public void add(Students std) {
            if (idx < listStd.length) {
                listStd[idx] = std;
                idx++;
            } else {
                System.out.println("Data is already full");
            }
        }
    \end{minted}
    \item Create method \texttt{display()} in class \texttt{SearchStudent}! This display() method will be used to print all students data available in this class. Pay attention on how we use \texttt{for loops} differently. Even so, the concepts is still the same
    \begin{minted}[autogobble,breaklines]{java}
        public void display() {
            for (Students students : listStd) {
                students.display();
                System.out.println("--------------------------------");
            }
        }
    \end{minted}
    \item Create method \texttt{FindSeqSearch} with integer as its return type. Then fill in the function with sequential search algorithm.
    \begin{minted}[autogobble,breaklines]{java}
        public int findSeqSearch(int search) {
            int potition = 1;
            for (int i = 0; i < listStd.length; i++) {
                if (listStd[i].nim == search) {
                    potition = i;
                    break;
                }
            }
            return potition;
        }
    \end{minted}
    \item Create method \texttt{displayPosition} with void as its return type. And write these following code as follows
    \begin{minted}[autogobble,breaklines]{java}
        public void showPotition(int x, int pos) {
            if (pos != 1) {
                System.out.println("Data : " + x + " is found in index-" + pos);
            } else {
                System.out.println("Data : " + x + "is not found");
            }
        }
    \end{minted}
    \item Create method \texttt{displayData} with void as its return type. And write these following code as follows
    \begin{minted}[autogobble,breaklines]{java}
        public void showData(int x, int pos) {
            if (pos != 1) {
                System.out.println("NIM \t : " + x);
                System.out.println("Name \t : " + listStd[pos].name);
                System.out.println("Age \t : " + listStd[pos].age);
                System.out.println("IPK \t : " + listStd[pos].gpa);
            } else {
                System.out.println("Data " + x + " is not found");
            }
        }
    \end{minted}
    \item Create a main class named \texttt{StudentsMain} and add main method as follows
    \begin{minted}[autogobble,breaklines]{java}
        public class StudentsMain {
            public static void main(String[] args) {
            
            }
        }
    \end{minted}
    \item In main method, instantiate an object in \texttt{SearchStudent} that consist of \texttt{5 Students}, then add all students object by calling \texttt{add} function in object \texttt{SearchStudent}
    \begin{minted}[autogobble,breaklines]{java}
        Scanner s = new Scanner(System.in);
        Scanner sl = new Scanner(System.in);

        SearchStudent data = new SearchStudent();
        int amountStudent = 5;

        System.out.println("--------------------------------");
        System.out.println("Input student data accordingly from samllest NIM");
        for (int i = 0; i < amountStudent; i++) {
            System.out.println("----------------");
            System.out.print("NIM\t: ");
            int nim = s.nextInt();
            System.out.print("Name\t: ");
            String name = sl.nextLine();
            System.out.print("Age\t: ");
            int age = s.nextInt();
            System.out.print("GPA\t: ");
            double gpa = s.nextDouble();

            Students std = new Students(nim, age, name, gpa);
            data.add(std);
        }
    \end{minted}
    \item Add method \texttt{display} to print all inserted data
    \begin{minted}[autogobble,breaklines]{java}
        System.out.println("--------------------------------");
        System.out.println("Entire Student Data");
        data.display();
    \end{minted}
    \item To search students by their NIM, create a search variable to hold input from user. Then call method FindSeqSearch with its parameter is the search variable we’ve declared before
    \begin{minted}[autogobble,breaklines]{java}
        System.out.println("________________________");
        System.out.println("________________________");
        System.out.print("Search student by NIM: ");
        int search = s.nextInt();
        System.out.println("Using Sequential Search");
        int potition = data.findSeqSearch(search);
    \end{minted}
    \item Call method displayPosition from class SearchStudent.
    \begin{minted}[autogobble,breaklines]{java}
        data.showPotition(search, potition);
    \end{minted} 
    \item Call method displayData from class SearchStudent
    \begin{minted}[autogobble,breaklines]{java}
        data.showData(search, potition);
    \end{minted}
    \item Run the program and see the result
    \begin{minted}[autogobble,breaklines]{java}
        package week7;

        public class Students {
            int nim, age;
            String name;
            double gpa;

            public Students(int nim, int age, String name, double gpa) {
                this.nim = nim;
                this.age = age;
                this.name = name;
                this.gpa = gpa;
            }

            public void display() {
                System.out.println("NIM =" + nim);
                System.out.println("Name =" + name);
                System.out.println("Age =" + age);
                System.out.println("GPA =" + gpa);
            }
        }
    \end{minted}
    \begin{minted}[autogobble,breaklines]{java}
        package week7;

        public class SearchStudent {
            Students[] listStd = new Students[5];
            int idx;
            public void add(Students std) {
                if (idx < listStd.length) {
                    listStd[idx] = std;
                    idx++;
                } else {
                    System.out.println("Data is already full");
                }
            }

            public void display() {
                for (Students students : listStd) {
                    students.display();
                    System.out.println("--------------------------------");
                }
            }

            public int findSeqSearch(int search) {
                int potition = 1;
                for (int i = 0; i < listStd.length; i++) {
                    if (listStd[i].nim == search) {
                        potition = i;
                        break;
                    }
                }
                return potition;
            }

            public void showPotition(int x, int pos) {
                if (pos != 1) {
                    System.out.println("Data : " + x + " is found in index-" + pos);
                } else {
                    System.out.println("Data : " + x + "is not found");
                }
            }

            public void showData(int x, int pos) {
                if (pos != 1) {
                    System.out.println("NIM \t : " + x);
                    System.out.println("Name \t : " + listStd[pos].name);
                    System.out.println("Age \t : " + listStd[pos].age);
                    System.out.println("IPK \t : " + listStd[pos].gpa);
                } else {
                    System.out.println("Data " + x + " is not found");
                }
            }
        }
    \end{minted}
    \begin{minted}[autogobble,breaklines]{java}
        package week7;

        import java.util.Scanner;

        public class StudentsMain {
            public static void main(String[] args) {
                Scanner s = new Scanner(System.in);
                Scanner sl = new Scanner(System.in);

                SearchStudent data = new SearchStudent();
                int amountStudent = 5;

                System.out.println("--------------------------------");
                System.out.println("Input student data accordingly from samllest NIM");
                for (int i = 0; i < amountStudent; i++) {
                    System.out.println("----------------");
                    System.out.print("NIM\t: ");
                    int nim = s.nextInt();
                    System.out.print("Name\t: ");
                    String name = sl.nextLine();
                    System.out.print("Age\t: ");
                    int age = s.nextInt();
                    System.out.print("GPA\t: ");
                    double gpa = s.nextDouble();

                    Students std = new Students(nim, age, name, gpa);
                    data.add(std);
                }

                System.out.println("--------------------------------");
                System.out.println("Entire Student Data");
                data.display();

                System.out.println("________________________");
                System.out.println("________________________");
                System.out.print("Search student by NIM: ");
                int search = s.nextInt();
                System.out.println("Using Sequential Search");
                int potition = data.findSeqSearch(search);

                data.showPotition(search, potition);

                data.showData(search, potition);

                s.close();
                sl.close();
            }
        }
    \end{minted}
\end{enumerate}

\subsubsection{Result}

\begin{minted}[autogobble,breaklines,linenos]{text}
    PS D:\Kuliah>  d:; cd 'd:\Kuliah'; & 'C:\Program Files\Java\jdk-18.0.2.1\bin\java.exe' '-XX:+ShowCodeDetailsInExceptionMessages' '-cp' 'C:\Users\ASUS\AppData\Roaming\Code\User\workspaceStorage\ce3fcb236261368a6cbd019dc8ddda8b\redhat.java\jdt_ws\Kuliah_28156aa7\bin' 'week7.StudentsMain' 
    --------------------------------
    Input student data accordingly from samllest NIM
    ----------------
    NIM     : 2017
    Name    : Dewi Lestari
    Age     : 23
    GPA     : 3.5
    ----------------
    NIM     : 2018
    Name    : Sinta Sanjaya
    Age     : 22
    GPA     : 4
    ----------------
    NIM     : 2019
    Name    : Danang Adi
    Age     : 22
    GPA     : 3.7
    ----------------
    NIM     : 2020
    Name    : Budi Prakarsa 
    Age     : 20
    GPA     : 2.9
    ----------------
    NIM     : 2021
    Name    : Vania Siti
    Age     : 20
    GPA     : 3
    --------------------------------
    Entire Student Data
    NIM =2017
    Name =Dewi Lestari
    Age =23
    GPA =3.5
    --------------------------------
    NIM =2018
    Name =Sinta Sanjaya
    Age =22
    GPA =4.0
    --------------------------------
    NIM =2019
    Name =Danang Adi
    Age =22
    GPA =3.7
    --------------------------------
    NIM =2020
    Name =Budi Prakarsa
    Age =20
    GPA =2.9
    --------------------------------
    NIM =2021
    Name =Vania Siti
    Age =20
    GPA =3.0
    --------------------------------
    ________________________
    ________________________
    Search student by NIM: 2017
    Using Sequential Search
    Data : 2017 is found in index-0
    NIM      : 2017
    Name     : Dewi Lestari
    Age      : 23
    IPK      : 3.5
\end{minted}

\subsubsection{Question}

\begin{enumerate}
    \item What is the difference of method \textbf{displayData} and \textbf{displayPosition} in \textbf{StudentSearch} class?
    \mbox{}\\
    \texttt{Answer: }
    \mbox{}\\
    displayData shows the data being search by index in the list of students. displayPosition shows the index of the searched data in the list.
    \item What is the function of break in this following program code?
    \begin{minted}[autogobble,breaklines]{java}
        if (listStd[i].nim == search) {
                potition = i;
                break;
            }
    \end{minted}
    \mbox{}\\
    \texttt{Answer: }
    \mbox{}\\
    breaking the loop once the data being search has been founded.
    \item If inserted NIM data is not sorted from smallest to biggest value, will the program encounter an error? Is the result still correct? Why is that?
    \mbox{}\\
    \texttt{Answer: }
    \mbox{}\\
    the search algorithm will output the wrong data because the search algorithm search sequentially trough the index.
\end{enumerate}

\subsection{Binary Search Method}

\subsubsection{Steps}

\begin{enumerate}
    \item in step 1.2.1 (Sequential search), create method FindBinarySearch with integer as its data type in class SearchStudent. Then declare the content of method FindBinarySearch with using binary search as its searching algorithm
    \item Call method FindBinarySearch from SearchStudent class in StudentsMain. Then call method
    displayPosition and displayData
    \item Run and see the result
\end{enumerate}

\subsubsection{Question}

\begin{enumerate}
    \item Show the program code in which runs the divide process
    \mbox{}\\
    \texttt{Answer: }
    \begin{minted}[autogobble,breaklines]{java}
        if (find == listStd[mid].nim) {
                return mid;
            } else if (listStd[mid].nim > find) {
                return findBinarySearch(find, left, mid - 1);
            } else {
                return findBinarySearch(find, mid + 1, right);
            }
    \end{minted}
    \item Show the program code in which runs the conquer process
    \mbox{}\\
    \texttt{Answer: }
    \begin{minted}[autogobble,breaklines]{java}
        int mid;
        if (right >= left) {
            mid = (left + right) / 2;
            if (find == listStd[mid].nim) {
                return mid;
            } else if (listStd[mid].nim > find) {
                return findBinarySearch(find, left, mid - 1);
            } else {
                return findBinarySearch(find, mid + 1, right);
            }
        }
        return -1;
    \end{minted}
    \item If inserted NIM data is not sorted, will the program crash? Why? If inserted NIM data is sorted from largest to smallest value (e.g 20215, 20214 20212, 20211,20210) and element being searched is 20210. How is the result of binary search? does it return the correct one? if not, then change the code so that the binary search executed properly
    \mbox{}\\
    \texttt{Answer: }
    \mbox{}\\
    the program will output an error if the data is not sorted. because the program works by comparing the value of the search item with the division of the divide conquer index.
    \item Modify program above so that the students amount inserted is matched with user input
    \mbox{}\\
    \texttt{Answer: }
    \mbox{}\\
    \begin{minted}[autogobble,breaklines]{java}
        Students[] listStd;
        int idx;

        public SearchStudent(int amount) {
            listStd = new Students[amount];
        }
    \end{minted}
\end{enumerate}

\subsection{Review Divide and Conquer}

\subsection{Assignments}

\begin{enumerate}
    \item Modify the searching program above with these requirements:
    \begin{enumerate}
        \item Before we search using binary search, we have to sort the data first. You can use whichever sorting algorithm that you are comfortable with
        \begin{minted}[autogobble,breaklines]{java}
            public int findBinarySearch(int find, int left, int right) {
                sort();
                int mid;
                if (right >= left) {
                    mid = (left + right) / 2;
                    if (find == listStd[mid].nim) {
                        if (mid == listStd.length - 1 || listStd[mid + 1].nim > find) {
                            return mid;
                        } else {
                            return findBinarySearch(find, mid + 1, right);
                        }
                    } else if (listStd[mid].nim > find) {
                        return findBinarySearch(find, left, mid - 1);
                    } else {
                        return findBinarySearch(find, mid + 1, right);
                    }
                }
                return -1;
            }

            private void sort() {
                for (int i = 0; i < listStd.length - 1; i++) {
                    for (int j = 0; j < listStd.length - i - 1; j++) {
                        if (listStd[j].nim > listStd[j+1].nim) {
                            Students temp = listStd[j];
                            listStd[j] = listStd[j+1];
                            listStd[j+1] = temp;
                        }
                    }
                }
            } 
        \end{minted} 
    \end{enumerate}
    \item Modify the searching above with these requirements:
    \begin{itemize}
        \item Search by student’s name with Sequential Search algorithm
        \begin{minted}[autogobble,breaklines]{java}
            public int findSeqSearch(String search) {
                int potition = -1;
                for (int i = 0; i < listStd.length; i++) {
                    if (listStd[i].name == search) {
                        potition = i;
                        break;
                    }
                }
                return potition;
            }
        \end{minted}
        \item How is the output of the program if there is any duplicate name?
        \item There is 2d array as follows:
        \mbox{}\\
        \begin{tabular}{|c|c|c|c|c|c|}
            \hline
            index  & 0  & 1  & 2  & 3   & 4  \\
            \hline
            0      & 45 & 78 & 7  & 200 & 80 \\
            \hline
            1      & 90 & 1  & 17 & 100 & 50 \\
            \hline
            2      & 21 & 2  & 40 & 18  & 65 \\
            \hline
        \end{tabular}
        \mbox{}\\
        Based on data above, create a program to search data in 2d array, which the data to be searched is defined by user input (using sequential search)
        \begin{minted}[autogobble,breaklines]{java}
            public int[] findInArr2D(int search, int[][] arr2d) {
                for (int i = 0; i < arr2d.length; i++) {
                    for (int j = 0; j < arr2d[i].length; j++) {
                        if (arr2d[i][j] == search) return new int[]{i, j};
                    }
                }
                return null;
            }
        \end{minted}
        \item There is a 1D array as follows:
        \mbox{}\\
        \begin{tabular}{|c|c|c|c|c|c|c|c|c|c|}
            \hline
            0  & 1  & 2  & 3 & 4  & 5  & 6  & 7  & 8 & 9 \\
            \hline
            12 & 17 & 2  & 1 & 70 & 50 & 90 & 17 & 2 & 90 \\
            \hline
        \end{tabular}
        \mbox{}\\
        Create a program to sort the array, search \& display the biggest value, and print the amount of biggest value available alongside with its position.
        \begin{minted}[autogobble,breaklines]{java}
            package Assignment;

            public class ArraySearch {
                private static void displayData(int[] data) {
                    for (int i = 0; i < data.length; i++) {
                        System.out.printf("%s ", data[i]);
                    }
                    System.out.println();
                }

                private static int[] sortAscending(int[] data) {
                    for (int i = 1; i < data.length; i++) {
                        int tmp = data[i];
                        int j = i - 1;
                            while (j >= 0 && data[j] > tmp) {
                                data[j + 1] = data[j];
                                j--;
                            }
                        data[j + 1] = tmp;
                    }
                    return data;
                }

                private static int[] sortDescending(int[] data) {
                    for (int i = 1; i < data.length; i++) {
                        int tmp = data[i];
                        int j = i - 1;
                            while (j >= 0 && data[j] < tmp) {
                            data[j + 1] = data[j];
                            j--;
                            }
                        data[j + 1] = tmp;
                    }
                    return data;
                }

                private static int[] findLargestValue(int[] data) {
                    int largestPos = 0;
                    int largest = data[largestPos];
                    for (int i = 1; i < data.length; i++) {
                        if (data[i] > largest) {
                            largestPos = i;
                            largest = data[largestPos];
                        }
                    }
                    return new int[]{largestPos, largest};
                }

                private static void displayLargestValue(int[] data) {
                    int[] sortedData = sortAscending(data);
                    int[] largestValue = findLargestValue(sortedData);
                    System.out.printf("Largest value position : %d\n", largestValue[0]);
                    System.out.printf("Largest value : %d\n", largestValue[1]);
                }

                public static void main(String[] args) {
                    int[] data = {5, 7, 4, 32, 6, 7, 89, 56, 3, 5, 7, 78, 3};
                    System.out.println("Unsorted data: ");
                    displayData(data);
                    
                    System.out.println("Sorted data (asc):");
                    int[] sortedDataAscending = sortAscending(data);
                    displayData(sortedDataAscending);

                    System.out.println("Sorted data (desc):");
                    int[] sortedDataDescending = sortDescending(data);
                    displayData(sortedDataDescending);

                    displayLargestValue(data);
                }
            }
        \end{minted}
        \begin{minted}[autogobble,breaklines,linenos]{text}
            Unsorted data:
            5 7 4 32 6 7 89 56 3 5 7 78 3
            Sorted data (asc):
            3 3 4 5 5 6 7 7 7 32 56 78 89
            Sorted data (desc):
            89 78 56 32 7 7 7 6 5 5 4 3 3
            Largest value position : 12
            Largest value : 89
        \end{minted}
    \end{itemize}
\end{enumerate}

\end{document}