\documentclass[12pt,titlepage]{article}
\usepackage[margin=1.25in]{geometry}
\usepackage{graphicx,amsmath,blindtext,minted}

%% Variables definition
\newcommand{\vSubject}{Project Management}
\newcommand{\vSubtitle}{Quiz 2}
\newcommand{\vName}{Muhammad Baihaqi Aulia Asy'ari}
\newcommand{\vNIM}{2241720145}
\newcommand{\vClass}{2I}
\newcommand{\vDepartment}{Information Technology}
\newcommand{\vStudyProgram}{D4 Informatics Engineering}

%% [START] Tikz related stuff
\usepackage{tikz}
\usetikzlibrary{svg.path,calc,shapes.geometric,shapes.misc}
\tikzstyle{terminator} = [rectangle, draw, text centered, rounded corners = 1em, minimum height=2em]
\tikzstyle{preparation} = [chamfered rectangle, chamfered rectangle sep=0.75em, draw, text centered, minimum height = 2em]
\tikzstyle{process} = [rectangle, draw, text centered, minimum height=2em]
\tikzstyle{decision} = [diamond, aspect=2, draw, text centered, minimum height=2em]
\tikzstyle{data}=[trapezium, draw, text centered, trapezium left angle=60, trapezium right angle=120, minimum height=2em]
\tikzstyle{connector} = [line width=0.25mm,->]
%% [END] Tikz related stuff

%% [START] Fancy header related stuff
\usepackage{fancyhdr}
\pagestyle{fancy}
\setlength{\headheight}{15pt} % compensate fancyhdr style
\fancyhead{}
\fancyfoot{}
\fancyfoot[L]{\thepage}
\fancyfoot[R]{\textit{\vSubject - \vSubtitle}}
\renewcommand{\footrulewidth}{0.4pt}% default is 0pt, overline for footer
%% [END] Fancy header related stuff

%% [START] Custom tabular command related stuff
\usepackage{tabularx}
\newcommand{\details}[2]{
    #1 & #2  \\
}
%% [END] Custom tabular command related stuff

%% [START] Figure related stuff
\newcommand{\image}[3][1]{
    \begin{figure}[h]
        \centering
        \includegraphics[#1]{#2}
        \caption{#3}
        \label{#3}
    \end{figure}
}
%% [END] Figure related stuff

%%
\usepackage{pgf-umlcd}

\renewcommand{\umldrawcolor}{black}
\renewcommand{\umlfillcolor}{white}
%%

%% [BEGIN] Custom enumerator
\usepackage{enumitem}
%% [END] Custom enumerator

%% [BEGIN] Paragraph indent
\usepackage{indentfirst}
%% [END] Paragraph indent

\usepackage{array}

\begin{document}
\begin{titlepage}
    \centering
    \vfill
    {\bfseries\LARGE
        \vSubject\\
        \vskip0.25cm
        \vSubtitle
    }
    \vfill
    \includegraphics[width=6cm]{images/polinema-logo.png}
    \vfill
    {
        \textbf{Group Members}\\
        \vspace{0.5cm}
        \begin{tabular}{l l}
            Dicha Zelianivan Arkana         & \textbf{2241720002} \\
            Davis Maulana Hermanto          & \textbf{2241720255} \\
            Muhammad Baihaqi Aulia Asy'ari  & \textbf{2241720145} \\
            Sri Kresna Maha Dewa            & \textbf{2241720244} \\
            Steven Christian Susanto        & \textbf{2241720003} \\
            Yanuar Thaif Chalil Candra      & \textbf{2241720004} \\            
        \end{tabular}
        \vskip0.5cm
        \textbf{Class}\\
        \vClass\\
        \vskip0.5cm
        \textbf{Department}\\
        \vDepartment\\
        \vskip0.5cm
        \textbf{Study Program}\\
        \vStudyProgram
    }
\end{titlepage}

\newpage

\section*{Quiz 2 - Project Management}
\subsection*{Question}
\noindent
Based on the project you planned at the beginning (the project you proposed during the Midterm Exam), answer the following questions:
\begin{enumerate}
    \item What will you do to manage Human Resources in your project. Explain in full!
    \item What do you do to manage the Communications in your project. Explain in full!
    \item Is there a connection between HR management and communication in implementing the software development project you are carrying out? Explain in detail!
\end{enumerate}
\subsection*{Answer}
\begin{enumerate}
    \item To ease personnel management, we plan to use a Responsibility Assignment Matrix (RAM) using RACI chart format based on the Work Breakdown Structure (WBS) we have previously made. This way, we can make sure everyone knows what they're supposed to be doing and balance the workload fairly and efficiently.
    \item When it comes to communication, we will develop a communication plan by identifying everyone involved and understanding what they need. We will create a detailed plan on how information will be shared. We will be providing regular updates, sharing documents, encouraging open communication, and in-person interaction. We will summarize all this information in reports to show how the project is going. Talking to everyone, solving problems, and making sure everyone knows what's happening are all important for steering our project in the right direction. 
    \item In our software development project, we closely integrate HR management and communication. By employing a Responsibility Assignment Matrix (RAM) based on the Work Breakdown Structure (WBS), we establish clear roles and ensure an equitable distribution of workload. This approach aligns human resources effectively with project tasks, promoting transparency and balance within the team.
\end{enumerate}

\newpage

\subsection*{Work Breakdown Structure}
\includegraphics[width=\textwidth]{images/figures/fig1.png}

\subsection*{RACI chart}
\resizebox{\textwidth}{!}{
    \begin{tabular}[c]{|l|*{6}{p{1.5cm}|}}
        \hline
        \textbf{Task/Role} & \textbf{PM} & \textbf{Dev 1} & \textbf{Dev 2} & \textbf{UI/UX} & \textbf{QA 1} & \textbf{QA 2} \\
        \hline
        Design Architecture: UX & A & C & C & R & I & I \\
        \hline
        Design Architecture: Wireframe & A & C & C & R & I & I \\
        \hline
        Design Architecture: Mapping UI & A & C & C & R & I & I \\
        \hline
        Web Structure: Frontend & A & C & R & I & I & I \\
        \hline
        Web Structure: Backend & A & R & C & I & I & I \\
        \hline
        Documentation & A & R & C & I & I & I\\
        \hline
        Control a scope creep & R & I & I & I & C & A \\
        \hline
        Change management processes & R & I & I & I & A & C \\
        \hline
        Quality Management: Unit Testing & A & C & C & I & I & R \\
        \hline
        Quality Management: System Testing & A & C & C & I & R & I \\
        \hline
        Lesson Learned & R & I & I & I & C & A \\
        \hline
        Project Report & R & I & I & I & A & C \\
        \hline
        Product Delivery & R & A & C & I & I & I \\
        \hline
        Project Close-out Meeting & R & C & A & I & I & I \\
        \hline
    \end{tabular}
}

\subsection*{Resource Allocation Histogram}
\includegraphics[width=\textwidth]{images/figures/fig2.jpg}

\end{document}