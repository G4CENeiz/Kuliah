\documentclass[12pt,titlepage]{article}
\usepackage[margin=1.25in]{geometry}
\usepackage{graphicx,amsmath,blindtext,minted}

%% Variables definition
\newcommand{\vSubject}{Object Oriented Programming}
\newcommand{\vSubtitle}{Inheritance}
\newcommand{\vName}{Muhammad Baihaqi Aulia Asy'ari}
\newcommand{\vNIM}{2241720145}
\newcommand{\vClass}{2I}
\newcommand{\vDepartment}{Information Technology}
\newcommand{\vStudyProgram}{D4 Informatics Engineering}

%% [START] Tikz related stuff
\usepackage{tikz}
\usetikzlibrary{svg.path,calc,shapes.geometric,shapes.misc}
\tikzstyle{terminator} = [rectangle, draw, text centered, rounded corners = 1em, minimum height=2em]
\tikzstyle{preparation} = [chamfered rectangle, chamfered rectangle sep=0.75em, draw, text centered, minimum height = 2em]
\tikzstyle{process} = [rectangle, draw, text centered, minimum height=2em]
\tikzstyle{decision} = [diamond, aspect=2, draw, text centered, minimum height=2em]
\tikzstyle{data}=[trapezium, draw, text centered, trapezium left angle=60, trapezium right angle=120, minimum height=2em]
\tikzstyle{connector} = [line width=0.25mm,->]
%% [END] Tikz related stuff

%% [START] Fancy header related stuff
\usepackage{fancyhdr}
\pagestyle{fancy}
\setlength{\headheight}{15pt} % compensate fancyhdr style
\fancyhead{}
\fancyfoot{}
\fancyfoot[L]{\thepage}
\fancyfoot[R]{\textit{\vSubject - \vSubtitle}}
\renewcommand{\footrulewidth}{0.4pt}% default is 0pt, overline for footer
%% [END] Fancy header related stuff

%% [START] Custom tabular command related stuff
\usepackage{tabularx}
\newcommand{\details}[2]{
    #1 & #2  \\
}
%% [END] Custom tabular command related stuff

%% [START] Figure related stuff
\newcommand{\image}[3][1]{
    \begin{figure}[h]
        \centering
        \includegraphics[#1]{#2}
        \caption{#3}
        \label{#3}
    \end{figure}
}
%% [END] Figure related stuff

%%
\usepackage{pgf-umlcd}

\renewcommand{\umldrawcolor}{black}
\renewcommand{\umlfillcolor}{white}
%%

%% [BEGIN] Custom enumerator
\usepackage{enumitem}
%% [END] Custom enumerator

%% [BEGIN] Paragraph indent
\usepackage{indentfirst}
%% [END] Paragraph indent

\begin{document}
\begin{titlepage}
    \centering
    \vfill
    {\bfseries\LARGE
        \vSubject\\
        \vskip0.25cm
        \vSubtitle
    }
    \vfill
    \includegraphics[width=6cm]{images/polinema-logo.png}
    \vfill
    {
        \textbf{Name}\\
        \vName\\
        \vskip0.5cm
        \textbf{NIM}\\
        \vNIM\\
        \vskip0.5cm
        \textbf{Class}\\
        \vClass\\
        \vskip0.5cm
        \textbf{Department}\\
        \vDepartment\\
        \vskip0.5cm
        \textbf{Study Program}\\
        \vStudyProgram
    }
\end{titlepage}

\newpage

\section{Experiment 1}

\texttt{Main.java}
\begin{minted}[autogobble,breaklines,linenos]{java}
    package experiment1;

    public class Main {
        public static void main(String[] args) {
            Manager manager = new Manager();
            manager.name = "Anu";
            manager.address = "Home";
            manager.age = 101;
            manager.gender = "Fe male";
            manager.salary = 3_000_000;
            manager.bonus = 1_000_000;
            manager.showManagerData();
            
            Staff staff = new Staff();
            staff.name = "Itu";
            staff.address = "Alone";
            staff.age = 42;
            staff.gender = "Fe male";
            staff.salary = 2_000_000;
            staff.overtime = 500_000;
            staff.paycut = 250_000;
            staff.showStaffData();
        }
    }

    class Employee {
        public String name, address, gender;
        public int age, salary;

        public Employee() {
        }

        public Employee(String name, String address, String gender, int age, int salary) {
            this.name = name;
            this.address = address;
            this.gender = gender;
            this.age = age;
            this.salary = salary;
        }

        public void showEmployeeData() {
            System.out.println(String.format("Name          : %s", name));
            System.out.println(String.format("Address       : %s", address));
            System.out.println(String.format("Gender        : %s", gender));
            System.out.println(String.format("Age           : %d", age));
            System.out.println(String.format("Salary        : %,d", salary));
        }
    }

    class Manager extends Employee {
        public int bonus;

        public Manager() {
        }

        public void showManagerData() {
            super.showEmployeeData();
            System.out.println(String.format("Bonus         : %,d", bonus));
            System.out.println(String.format("Total Salary  : %,d", super.salary+bonus));
        }
    }

    class Staff extends Employee {
        public int overtime, paycut;
        
        public Staff() {
        }
        
        public Staff(String name, String address, String gender, int age, int salary, int overtime, int paycut) {
            super(name, address, gender, age, salary);
            this.overtime = overtime;
            this.paycut = paycut;
        }
        
        public void showStaffData() {
            super.showEmployeeData();
            System.out.println(String.format("Overtime      : %,d", overtime));
            System.out.println(String.format("Paycut        : %,d", paycut));
            System.out.println(String.format("Total Salary  : %,d", super.salary+overtime-paycut));
        }
    }
\end{minted}

\texttt{Terminal}
\begin{minted}[autogobble,breaklines,linenos]{text}
    PS D:\Kuliah>  d:; cd 'd:\Kuliah'; & 'C:\Program Files\Java\jdk-18.0.2.1\bin\java.exe' '-XX:+ShowCodeDetailsInExceptionMessages' '-cp' 'C:\Users\G4CE-PC\AppData\Roaming\Code\User\workspaceStorage\ 80d97a47d24665dc0bce7ab1e048ecbd\redhat.java\jdt_ws\ Kuliah_28156aa7\bin' 'experiment1.Main' 
    Name          : Anu
    Address       : Home
    Gender        : Fe male
    Age           : 101
    Salary        : 3,000,000
    Bonus         : 1,000,000
    Total Salary  : 4,000,000
    Name          : Itu
    Address       : Alone
    Gender        : Fe male
    Age           : 42
    Salary        : 2,000,000
    Overtime      : 500,000
    Paycut        : 250,000
    Total Salary  : 2,250,000
\end{minted}

\newpage

\subsection{Question}
\begin{enumerate}
    \item Sebutkan class mana yang termasuk super class dan sub class dari percobaan 1 diatas!
    \item Kata kunci apakah yang digunakan untuk menurunkan suatu class ke class yang lain?
    \item Perhatikan kode program pada class Manager, atribut apa saja yang dimiliki oleh class tersebut? Sebutkan atribut mana saja yang diwarisi dari class Karyawan!
    \item Jelaskan kata kunci super pada potongan program dibawah ini yang terdapat pada class Manager!
    \begin{minted}[autogobble,breaklines]{java}
        System.out.println(String.format("Total Salary  : %,d", super.salary+bonus)); 
    \end{minted}
    \item Program pada percobaan 1 diatas termasuk dalam jenis inheritance apa? Jelaskan alasannya!
\end{enumerate}

\subsection{Answer}
\begin{enumerate}
    \item The Employee class is a superclass and The Manager and Staff class are the subclass.
    \item extends
    \item name, address, gender, age, and salary are inherited from the Employee class. The bonus attribute is the only attribute that is not inherited from the Employee class.
    \item The super keyword refer to the superclass attributes or methods. In this case it is used to refer to the salary attribute of the superclass.
    \item It is a Hierarchical Inheritance because the superclass has more than one subclass.
\end{enumerate}

\newpage

\section{Experiment 2}

\texttt{Main.java}
\begin{minted}[autogobble,breaklines,linenos]{java}
    package experiment2;

    public class Main {
        public static void main(String[] args) {
            PermanentStaff permanentStaff = new PermanentStaff("Anu", "Home", "Fe Male", 34, 2_000_000, 250_000, 200_000, "2A", 100_000);
            permanentStaff.showPermanentStaffData();

            DailyStaff dailyStaff = new DailyStaff("Itu", "Alone", "Fe Male", 1738, 10_000, 100_000, 50_000, 100);
            dailyStaff.showDailyStaffData();
        }
    }

    class Employee {
        public String name, address, gender;
        public int age, salary;

        public Employee() {
        }

        public Employee(String name, String address, String gender, int age, int salary) {
            this.name = name;
            this.address = address;
            this.gender = gender;
            this.age = age;
            this.salary = salary;
        }

        public void showEmployeeData() {
            System.out.println(String.format("Name              : %s", name));
            System.out.println(String.format("Address           : %s", address));
            System.out.println(String.format("Gender            : %s", gender));
            System.out.println(String.format("Age               : %d", age));
            System.out.println(String.format("Salary            : %,d", salary));
        }
    }

    class Manager extends Employee {
        public int bonus;

        public Manager() {
        }

        public void showManagerData() {
            super.showEmployeeData();
            System.out.println(String.format("Bonus             : %,d", bonus));
            System.out.println(String.format("Total Salary      : %,d", super.salary+bonus));
        }
    }

    class Staff extends Employee {
        public int overtime, paycut;
        
        public Staff() {
        }
        
        public Staff(String name, String address, String gender, int age, int salary, int overtime, int paycut) {
            super(name, address, gender, age, salary);
            this.overtime = overtime;
            this.paycut = paycut;
        }
        
        public void showStaffData() {
            super.showEmployeeData();
            System.out.println(String.format("Overtime          : %,d", overtime));
            System.out.println(String.format("Paycut            : %,d", paycut));
            System.out.println(String.format("Total Salary      : %,d", super.salary+overtime-paycut));
        }
    }

    class PermanentStaff extends Staff {
        public String bracket;
        public int insurance;
        
        public PermanentStaff() {
        }
        
        public PermanentStaff(String name, String address, String gender, int age, int salary, int overtime, int paycut, String bracket, int insurance) {
            super(name, address, gender, age, salary, overtime, paycut);
            this.bracket = bracket;
            this.insurance = insurance;
        }
        
        public void showPermanentStaffData() {
            String bar = "================================";
            String template = String.format("%%%ds%%-%ds%%%ds", 16,"Permanent Staff Data".length(), 16);
            System.out.println(String.format(template, bar,"Permanent Staff Data", bar));
            super.showStaffData();
            System.out.println(String.format("Bracket           : %s", bracket));
            System.out.println(String.format("Insurance         : %,d", insurance));
            System.out.println(String.format("Net Salary        : %,d", super.salary+overtime-paycut-insurance));
        }
    }

    class DailyStaff extends Staff {
        public int totalWorkHours;

        public DailyStaff() {
        }

        public DailyStaff(String name, String address, String gender, int age, int salary, int overtime, int paycut, int totalWorkHours) {
            super(name, address, gender, age, salary, overtime, paycut);
            this.totalWorkHours = totalWorkHours;
        }
        public void showDailyStaffData() {
            String bar = "================================";
            String template = String.format("%%%ds%%-%ds%%%ds", 16,"Daily Staff Data".length(), 16);
            System.out.println(String.format(template, bar,"Daily Staff Data", bar));
            super.showStaffData();
            System.out.println(String.format("Total Work Hours  : %d", totalWorkHours));
            System.out.println(String.format("Net Salary        : %,d", super.salary*totalWorkHours+overtime-paycut));
        }
    }
\end{minted}

\newpage

\texttt{Terminal}
\begin{minted}[autogobble,breaklines,linenos]{text}
    PS D:\Kuliah>  & 'C:\Program Files\Java\jdk-18.0.2.1\bin\java.exe' '-XX:+ShowCodeDetailsInExceptionMessages' '-cp' 'C:\Users\G4CE-PC\AppData\Roaming\Code\User\workspaceStorage\80d97a47d24665dc0bce7ab1e048ecbd\redhat.java\jdt_ws\Kuliah_28156aa7\bin' 'experiment2.Main'
    ================================Permanent Staff Data================================
    Name              : Anu      
    Address           : Home     
    Gender            : Fe Male  
    Age               : 34       
    Salary            : 2,000,000
    Overtime          : 250,000
    Paycut            : 200,000
    Total Salary      : 2,050,000
    Bracket           : 2A
    Insurance         : 100,000
    Net Salary        : 1,950,000
    ================================Daily Staff Data================================
    Name              : Itu
    Address           : Alone
    Gender            : Fe Male
    Age               : 1738
    Salary            : 10,000
    Overtime          : 100,000
    Paycut            : 50,000
    Total Salary      : 60,000
    Total Work Hours  : 100
    Net Salary        : 1,050,000
\end{minted}

\newpage

\subsection{Question}
\begin{enumerate}
    \item Berdasarkan class diatas manakah yang termasuk single inheritance dan mana yang termasuk multilevel inheritance?
    \item Perhatikan kode program class StaffTetap dan StaffHarian, atribut apa saja yang dimiliki oleh class tersebut? Sebutkan atribut mana saja yang diwarisi dari class Staff!
    \item Apakah fungsi potongan program berikut pada class StaffHarian
    \begin{minted}[autogobble,breaklines]{java}
        super(name, address, gender, age, salary, overtime, paycut);
    \end{minted}
    \item Apakah fungsi potongan program berikut pada class StaffHarian
    \begin{minted}[autogobble,breaklines]{java}
        super.showStaffData();
    \end{minted}
    \item Perhatikan kode program dibawah ini yang terdapat pada class StaffTetap
    \begin{minted}[autogobble,breaklines]{java}
        System.out.println(String.format("Net Salary        : %,d", super.salary+overtime-paycut-insurance));
    \end{minted}
    Terlihat dipotongan program diatas atribut gaji, lembur dan potongan dapat diakses langsung. Kenapa hal ini bisa terjadi dan bagaimana class StaffTetap memiliki atribut gaji, lembur, dan potongan padahal dalam class tersebut tidak dideklarasikan atribut gaji, lembur, dan potongan?
\end{enumerate}

\subsection{Answer}
\begin{enumerate}
    \item Manager and Staff class are single inheritance, while DailyStaff and permanentStaff are multilevel inheritance.
    \item Both class inherited the name, address, gender, age, salary, overtime, paycut from the Employee class. The DailyStaff has the attribute totalWorkHours of their own and the permanentStaff has the attributes bracket, and insurance.
    \item It is used to construct the superclass with the required attributes of the superclass from the subclass constructor parameter.
    \item It is used to call a method from the superclass.
    \item Because the subclass also inherit the attribute and the attribute it self is public thus making it accessable to the subclass.
\end{enumerate}

\newpage

\section{Assignment}
\noindent
Buatlah sebuah program dengan konsep pewarisan seperti pada class diagram berikut ini. Kemudian buatlah instansiasi objek untuk menampilkan data pada class Mac, Windows dan Pc!.
\begin{center}
    \includegraphics[width=.95\textwidth]{images/figures/fig1.png}
\end{center}

\texttt{Main.java}
\begin{minted}[autogobble,breaklines,linenos]{java}
    package assignment;

    public class Main {
        public static void main(String[] args) {
            PC pc = new PC("Dell", 3_600, 16, "AMD Ryzen 5 3500x", 24);
            pc.showPC();
            Mac mac = new Mac("Apple", 3_200, 8, "M1", "udisclosed information", "XProtect");
            mac.showMac();
            Windows windows = new Windows("ROG Strix G15 G513RM", 3_200, 8, "AMD Ryzen 7 6800H", "56WHrs", "NVIDIA GeForce RTX 3060 Laptop GPU");
            windows.showWindows();
        }
    }

    class Computer {
        public String brand;
        public int coreClock;
        public int ramSize;
        public String processorName;

        public Computer() {
        }

        public Computer(String brand, int coreClock, int ramSize, String processorName) {
            this.brand = brand;
            this.coreClock = coreClock;
            this.ramSize = ramSize;
            this.processorName = processorName;
        }

        public void showData() {
            System.out.println(String.format("Brand         : %s", brand));
            System.out.println(String.format("Core Clock    : %,d Mhz", coreClock));
            System.out.println(String.format("RAM           : %d Gb", ramSize));
            System.out.println(String.format("Processor Name: %s", processorName));
        }
    }

    class PC extends Computer {
        public int monitorSize;
        
        public PC() {
        }
        
        public PC(String brand, int coreClock, int ramSize, String processorName, int monitorSize) {
            super(brand, coreClock, ramSize, processorName);
            this.monitorSize = monitorSize;
        }
        
        public void showPC() {
            super.showData();
            System.out.println(String.format("Monitor Size  : %d inch", monitorSize));
        }
    }

    class Laptop extends Computer {
        public String batteryType;
        
        public Laptop() {
        }
        
        public Laptop(String brand, int coreClock, int ramSize, String processorName, String batteryType) {
            super(brand, coreClock, ramSize, processorName);
            this.batteryType = batteryType;
        }
        
        public void showLaptop() {
            super.showData();
            System.out.println(String.format("Battery Type  : %s", batteryType));
        }
    }

    class Mac extends Laptop {
        public String security;
        
        public Mac() {
        }
        
        public Mac(String brand, int coreClock, int ramSize, String processorName, String batteryType, String security) {
            super(brand, coreClock, ramSize, processorName, batteryType);
            this.security = security;
        }
        
        public void showMac() {
            super.showLaptop();
            System.out.println(String.format("Security      : %s", security));
        }
    }

    class Windows extends Laptop {
        public String features;
        
        public Windows() {
        }
        
        public Windows(String brand, int coreClock, int ramSize, String processorName, String batteryType, String features) {
            super(brand, coreClock, ramSize, processorName, batteryType);
            this.features = features;
        }
        
        public void showWindows() {
            super.showLaptop();
            System.out.println(String.format("Features      : %s", features));
        }
    }
\end{minted}

\texttt{Terminal}
\begin{minted}[autogobble,breaklines,linenos]{text}
    PS D:\Kuliah>  d:; cd 'd:\Kuliah'; & 'C:\Program Files\Java\jdk-18.0.2.1\bin\java.exe' '-XX:+ShowCodeDetailsInExceptionMessages' '-cp' 'C:\Users\G4CE-PC\AppData\Roaming\Code\User\workspaceStorage\ 80d97a47d24665dc0bce7ab1e048ecbd\redhat.java\jdt_ws\ Kuliah_28156aa7\bin' 'assignment.Main' 
    Brand         : Dell
    Core Clock    : 3,600 Mhz
    RAM           : 16 Gb
    Processor Name: AMD Ryzen 5 3500x
    Monitor Size  : 24 inch

    Brand         : Apple
    Core Clock    : 3,200 Mhz
    RAM           : 8 Gb
    Processor Name: M1
    Battery Type  : udisclosed information
    Security      : XProtect

    Brand         : ROG Strix G15 G513RM
    Core Clock    : 3,200 Mhz
    RAM           : 8 Gb
    Processor Name: AMD Ryzen 7 6800H
    Battery Type  : 56WHrs
    Features      : NVIDIA GeForce RTX 3060 Laptop GPU
\end{minted}

\end{document}