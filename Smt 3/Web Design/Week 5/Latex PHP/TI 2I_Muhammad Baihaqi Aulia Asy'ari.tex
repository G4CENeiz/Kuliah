\documentclass[12pt,titlepage]{article}
\usepackage[margin=1.25in]{geometry}
\usepackage{graphicx,amsmath,blindtext,minted}

%% Variables definition
\newcommand{\vSubject}{Web Design and Development}
\newcommand{\vSubtitle}{PHP}
\newcommand{\vName}{Muhammad Baihaqi Aulia Asy'ari}
\newcommand{\vNIM}{2241720145}
\newcommand{\vClass}{2I}
\newcommand{\vDepartment}{Information Technology}
\newcommand{\vStudyProgram}{D4 Informatics Engineering}

%% [START] Tikz related stuff
\usepackage{tikz}
\usetikzlibrary{svg.path,calc,shapes.geometric,shapes.misc}
\tikzstyle{terminator} = [rectangle, draw, text centered, rounded corners = 1em, minimum height=2em]
\tikzstyle{preparation} = [chamfered rectangle, chamfered rectangle sep=0.75em, draw, text centered, minimum height = 2em]
\tikzstyle{process} = [rectangle, draw, text centered, minimum height=2em]
\tikzstyle{decision} = [diamond, aspect=2, draw, text centered, minimum height=2em]
\tikzstyle{data}=[trapezium, draw, text centered, trapezium left angle=60, trapezium right angle=120, minimum height=2em]
\tikzstyle{connector} = [line width=0.25mm,->]
%% [END] Tikz related stuff

%% [START] Fancy header related stuff
\usepackage{fancyhdr}
\pagestyle{fancy}
\setlength{\headheight}{15pt} % compensate fancyhdr style
\fancyhead{}
\fancyfoot{}
\fancyfoot[L]{\thepage}
\fancyfoot[R]{\textit{\vSubject - \vSubtitle}}
\renewcommand{\footrulewidth}{0.4pt}% default is 0pt, overline for footer
%% [END] Fancy header related stuff

%% [START] Custom tabular command related stuff
\usepackage{tabularx}
\newcommand{\details}[2]{
    #1 & #2  \\
}
%% [END] Custom tabular command related stuff

%% [START] Figure related stuff
\newcommand{\image}[3][1]{
    \begin{figure}[h]
        \centering
        \includegraphics[#1]{#2}
        \caption{#3}
        \label{#3}
    \end{figure}
}
%% [END] Figure related stuff

%%
\usepackage{pgf-umlcd}

\renewcommand{\umldrawcolor}{black}
\renewcommand{\umlfillcolor}{white}
%%

%% [BEGIN] Custom enumerator
\usepackage{enumitem}
%% [END] Custom enumerator

%% [BEGIN] Paragraph indent
\usepackage{indentfirst}
%% [END] Paragraph indent

\begin{document}
\begin{titlepage}
    \centering
    \vfill
    {\bfseries\LARGE
        \vSubject\\
        \vskip0.25cm
        \vSubtitle
    }
    \vfill
    \includegraphics[width=6cm]{images/polinema-logo.png}
    \vfill
    {
        \textbf{Name}\\
        \vName\\
        \vskip0.5cm
        \textbf{NIM}\\
        \vNIM\\
        \vskip0.5cm
        \textbf{Class}\\
        \vClass\\
        \vskip0.5cm
        \textbf{Department}\\
        \vDepartment\\
        \vskip0.5cm
        \textbf{Study Program}\\
        \vStudyProgram
    }
\end{titlepage}

\newpage

\begin{enumerate}[label*=\arabic*.]
    \section*{Practicum 1  - Variable and Constants}
    \item variable can be declared in various ways with various data type.
    \section*{Practicum 2  - Data Type Usage}
    \item variable can store multiple data type and there are various data type you can identify with var\textunderscore dump
    \section*{Practicum 3  - Operators in PHP}
    \item -
    \begin{enumerate}[label*=\arabic*.]
        \item - \\ \includegraphics[width=.8\textwidth]{images/figures/fig1.png} \\ there are various aritmathic operators in php including specialized squaring operators which not many languages have. 
        \item - \\ \includegraphics[width=.8\textwidth]{images/figures/fig2.png} \\ you can compare the value of variable.
        \item - \\ \includegraphics[width=.8\textwidth]{images/figures/fig3.png} \\ logical operators can be used as boolean condition.
        \item - \\ \includegraphics[width=.8\textwidth]{images/figures/fig4.png} \\ assignment operators can simplify the usage of aritmathic operators. 
        \item - \\ \includegraphics[width=.8\textwidth]{images/figures/fig5.png} \\ there are operators to identify whether or not a variable is identical to another. 
        \item - \\ \includegraphics[width=.8\textwidth]{images/figures/fig6.png} \\ \includegraphics[width=.8\textwidth]{images/figures/fig7.png} \\ -
    \end{enumerate}
    \section*{Practicum 4  - PHP control structure}
    \item -
    \begin{enumerate}[label*=\arabic*.]
        \item - \\ \includegraphics[width=.8\textwidth]{images/figures/fig8.png} \\ the control structure goes over the condition of each if clause.
    \end{enumerate}
\end{enumerate}


\end{document}