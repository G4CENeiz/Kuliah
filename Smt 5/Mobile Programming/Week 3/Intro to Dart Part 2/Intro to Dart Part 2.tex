\documentclass[12pt,titlepage]{article}
\usepackage[margin=1.25in]{geometry}
\usepackage{graphicx,amsmath,blindtext,minted}

% Variables definition
\newcommand{\vSubject}{Mobile Programming}
\newcommand{\vSubtitle}{Intro to Dart - Part 2}
\newcommand{\vName}{Muhammad Baihaqi Aulia Asy'ari}
\newcommand{\vNIM}{2241720145}
\newcommand{\vClass}{2I}
\newcommand{\vDepartment}{Information Technology}
\newcommand{\vStudyProgram}{D4 Informatics Engineering}

% [START] Tikz related stuff
\usepackage{tikz}
\usetikzlibrary{svg.path,calc,shapes.geometric,shapes.misc}
\tikzstyle{terminator} = [rectangle, draw, text centered, rounded corners = 1em, minimum height=2em]
\tikzstyle{preparation} = [chamfered rectangle, chamfered rectangle sep=0.75em, draw, text centered, minimum height = 2em]
\tikzstyle{process} = [rectangle, draw, text centered, minimum height=2em]
\tikzstyle{decision} = [diamond, aspect=2, draw, text centered, minimum height=2em]
\tikzstyle{data}=[trapezium, draw, text centered, trapezium left angle=60, trapezium right angle=120, minimum height=2em]
\tikzstyle{connector} = [line width=0.25mm,->]
% [END] Tikz related stuff

% [START] Fancy header related stuff
\usepackage{fancyhdr}
\pagestyle{fancy}
\setlength{\headheight}{15pt} 
% compensate fancyhdr style
\fancyhead{}
\fancyfoot{}
\fancyfoot[L]{\thepage}
\fancyfoot[R]{\textit{\vSubject\space - \vSubtitle}}
\renewcommand{\footrulewidth}{0.4pt}
% default is 0pt, overline for footer
% [END] Fancy header related stuff

% [START] Custom tabular command related stuff
\usepackage{tabularx}
\newcommand{\details}[2]{
    #1 
}
% [END] Custom tabular command related stuff

% [START] Figure related stuff
\newcommand{\image}[3][1]{
    \begin{figure}[h]
        \centering
        \includegraphics[#1]{#2}
        \caption{#3}
        \label{#3}
    \end{figure}
}
% [END] Figure related stuff

%
\usepackage{pgf-umlcd}

\renewcommand{\umldrawcolor}{black}
\renewcommand{\umlfillcolor}{white}
%

% [BEGIN] Custom enumerator
\usepackage{enumitem}
% [END] Custom enumerator

% [BEGIN] Paragraph indent
\usepackage{indentfirst}
% [END] Paragraph indent

% [BEGIN] URL
\usepackage{hyperref}
\hypersetup{
    colorlinks=true,
    linkcolor=blue,
    filecolor=magenta,      
    urlcolor=cyan,
    pdftitle={Overleaf Example},
    pdfpagemode=FullScreen,
    }

\urlstyle{same}
% [END] URL

\begin{document}
\begin{titlepage}
    \centering
    \vfill
    {\bfseries\LARGE
        \vSubject\\
        \vskip0.25cm
        \vSubtitle
    }
    \vfill
    \includegraphics[width=6cm]{images/polinema-logo.png}
    \vfill
    {
        \textbf{Name}\\
        \vName\\
        \vskip0.5cm
        \textbf{NIM}\\
        \vNIM\\
        \vskip0.5cm
        \textbf{Class}\\
        \vClass\\
        \vskip0.5cm
        \textbf{Department}\\
        \vDepartment\\
        \vskip0.5cm
        \textbf{Study Program}\\
        \vStudyProgram
    }
\end{titlepage}

\newpage

\section*{Practicum 1}
\subsection*{Step 1}
\begin{minted}[autogobble,breaklines,linenos]{dart}
    void main() {
      String test = "test2";
      if (test == "test1") {
        print("Test1");
      } else If (test == "test2") {
        print("Test2");
      } Else {
        print("Something else");
      }
      if (test == "test2") print("Test2 again");
    }
\end{minted}

\subsection*{Step 2}
\includegraphics[width=.9\textwidth]{images/figures/Screenshot 2024-09-17 130533.png} \\
the compiler throw an error on line 19 and 5 due to captilization on the syntax 'If' and 'Else'

\subsection*{Step 3}
\begin{minted}[autogobble,breaklines,linenos]{dart}
      void main() {
        String test = "test2";
        if (test == "test1") {
          print("Test1");
        } else If (test == "test2") {
          print("Test2");
        } Else {
          print("Something else");
        }
      
        if (test == "test2") print("Test2 again");
        
        String test = "true";
        if (test) {
          print("Kebenaran");
        }
      }
\end{minted}

\includegraphics[width=.85\textwidth]{images/figures/Screenshot 2024-09-17 131635.png} \\ 
It throw an error becuase of the previous code and the new added code. here's the correction for the code \\

\begin{minted}[autogobble,breaklines,linenos]{dart}
      void main() {
        String test = "test2";
        if (test == "test1") {
          print("Test1");
        } else if (test == "test2") {
          print("Test2");
        } else {
          print("Something else");
        }
      
        if (test == "test2") print("Test2 again");
        
        test = "true";
        if (test == "true") {
          print("Kebenaran");
        }
      }
\end{minted}

\newpage

\section*{Practicum 2}
\subsection*{Step 1}
\begin{minted}[autogobble,breaklines,linenos]{dart}
      void main() {
        while (counter < 33) {
          print(counter);
          counter++;
        }
      }
\end{minted}
\subsection*{Step 2}
\includegraphics[width=.9\textwidth]{images/figures/Screenshot 2024-09-17 134836.png} \\ 
it throw errors because the variable counter has not yet to be defined. here is the correction for the code example.
\begin{minted}[autogobble,breaklines,linenos]{dart}
  void main() {
    int counter = 0;
    while (counter < 33) {
      print(counter);
      counter++;
    }
  }
\end{minted}
\subsection*{Step 3}
it now prints to the condition of the do while

\newpage

\section*{Practicum 3}
\subsection*{Step 1}
\begin{minted}[autogobble,breaklines,linenos]{dart}
  void main() {
    for (Index = 10; index < 27; index++) {
      print(Index);
    }
  }
\end{minted}

\subsection*{Step 2}
\includegraphics[width=.9\textwidth]{images/figures/Screenshot 2024-09-18 090614.png} \\
it throws errors because the variable 'Index' has not been defined and mismatch of the variable name being used. here is the correction for the code example.
\begin{minted}[autogobble,breaklines,linenos]{dart}
  void main() {
    for (int index = 10; index < 27; index) {
      print(index);
    }
  }
\end{minted}
\subsection*{Step 3}
\includegraphics[width=.9\textwidth]{images/figures/Screenshot 2024-09-18 091416.png} \\
the code throws so much errors it needs its own page. here is the correction for the code example.
\begin{minted}[autogobble,breaklines,linenos]{dart}
  void main() {
    for (int index = 10; index < 27; index++) {
      print(index);
      if (index == 21) break;
      else if (index > 1 || index < 7) continue;
      print(index);
    }
  }
\end{minted}
the loop now breaks after the index turns to 21 and the second call to print never get to be run because the 'else if' condition is always true for every number of the variable 'index' as it skips the step with the continue keyword.

\newpage

\section*{Practicum Assignment}
\begin{minted}[autogobble,breaklines,linenos]{dart}
  void main() {
    int limit = 201;
    for (int i = 0; i < limit; i++) {
      if (i != 1) {
        if (i > 0 && i < 4) {
          // print(i);
          print("Muhammad Baihaqi Aulia Asy'ari - 2241720145");
        } else if (i % 2 != 0 && i % 3 != 0 && i % 5 != 0 && i % 7 != 0) {
          // print(i);
          print("Muhammad Baihaqi Aulia Asy'ari - 2241720145");
        }
      }
    }
  }
\end{minted}

\end{document}
